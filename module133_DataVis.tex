% Options for packages loaded elsewhere
\PassOptionsToPackage{unicode}{hyperref}
\PassOptionsToPackage{hyphens}{url}
\PassOptionsToPackage{dvipsnames,svgnames,x11names}{xcolor}
%
\documentclass[
  letterpaper,
  DIV=11,
  numbers=noendperiod]{scrartcl}

\usepackage{amsmath,amssymb}
\usepackage{iftex}
\ifPDFTeX
  \usepackage[T1]{fontenc}
  \usepackage[utf8]{inputenc}
  \usepackage{textcomp} % provide euro and other symbols
\else % if luatex or xetex
  \usepackage{unicode-math}
  \defaultfontfeatures{Scale=MatchLowercase}
  \defaultfontfeatures[\rmfamily]{Ligatures=TeX,Scale=1}
\fi
\usepackage{lmodern}
\ifPDFTeX\else  
    % xetex/luatex font selection
\fi
% Use upquote if available, for straight quotes in verbatim environments
\IfFileExists{upquote.sty}{\usepackage{upquote}}{}
\IfFileExists{microtype.sty}{% use microtype if available
  \usepackage[]{microtype}
  \UseMicrotypeSet[protrusion]{basicmath} % disable protrusion for tt fonts
}{}
\makeatletter
\@ifundefined{KOMAClassName}{% if non-KOMA class
  \IfFileExists{parskip.sty}{%
    \usepackage{parskip}
  }{% else
    \setlength{\parindent}{0pt}
    \setlength{\parskip}{6pt plus 2pt minus 1pt}}
}{% if KOMA class
  \KOMAoptions{parskip=half}}
\makeatother
\usepackage{xcolor}
\setlength{\emergencystretch}{3em} % prevent overfull lines
\setcounter{secnumdepth}{-\maxdimen} % remove section numbering
% Make \paragraph and \subparagraph free-standing
\makeatletter
\ifx\paragraph\undefined\else
  \let\oldparagraph\paragraph
  \renewcommand{\paragraph}{
    \@ifstar
      \xxxParagraphStar
      \xxxParagraphNoStar
  }
  \newcommand{\xxxParagraphStar}[1]{\oldparagraph*{#1}\mbox{}}
  \newcommand{\xxxParagraphNoStar}[1]{\oldparagraph{#1}\mbox{}}
\fi
\ifx\subparagraph\undefined\else
  \let\oldsubparagraph\subparagraph
  \renewcommand{\subparagraph}{
    \@ifstar
      \xxxSubParagraphStar
      \xxxSubParagraphNoStar
  }
  \newcommand{\xxxSubParagraphStar}[1]{\oldsubparagraph*{#1}\mbox{}}
  \newcommand{\xxxSubParagraphNoStar}[1]{\oldsubparagraph{#1}\mbox{}}
\fi
\makeatother

\usepackage{color}
\usepackage{fancyvrb}
\newcommand{\VerbBar}{|}
\newcommand{\VERB}{\Verb[commandchars=\\\{\}]}
\DefineVerbatimEnvironment{Highlighting}{Verbatim}{commandchars=\\\{\}}
% Add ',fontsize=\small' for more characters per line
\usepackage{framed}
\definecolor{shadecolor}{RGB}{241,243,245}
\newenvironment{Shaded}{\begin{snugshade}}{\end{snugshade}}
\newcommand{\AlertTok}[1]{\textcolor[rgb]{0.68,0.00,0.00}{#1}}
\newcommand{\AnnotationTok}[1]{\textcolor[rgb]{0.37,0.37,0.37}{#1}}
\newcommand{\AttributeTok}[1]{\textcolor[rgb]{0.40,0.45,0.13}{#1}}
\newcommand{\BaseNTok}[1]{\textcolor[rgb]{0.68,0.00,0.00}{#1}}
\newcommand{\BuiltInTok}[1]{\textcolor[rgb]{0.00,0.23,0.31}{#1}}
\newcommand{\CharTok}[1]{\textcolor[rgb]{0.13,0.47,0.30}{#1}}
\newcommand{\CommentTok}[1]{\textcolor[rgb]{0.37,0.37,0.37}{#1}}
\newcommand{\CommentVarTok}[1]{\textcolor[rgb]{0.37,0.37,0.37}{\textit{#1}}}
\newcommand{\ConstantTok}[1]{\textcolor[rgb]{0.56,0.35,0.01}{#1}}
\newcommand{\ControlFlowTok}[1]{\textcolor[rgb]{0.00,0.23,0.31}{\textbf{#1}}}
\newcommand{\DataTypeTok}[1]{\textcolor[rgb]{0.68,0.00,0.00}{#1}}
\newcommand{\DecValTok}[1]{\textcolor[rgb]{0.68,0.00,0.00}{#1}}
\newcommand{\DocumentationTok}[1]{\textcolor[rgb]{0.37,0.37,0.37}{\textit{#1}}}
\newcommand{\ErrorTok}[1]{\textcolor[rgb]{0.68,0.00,0.00}{#1}}
\newcommand{\ExtensionTok}[1]{\textcolor[rgb]{0.00,0.23,0.31}{#1}}
\newcommand{\FloatTok}[1]{\textcolor[rgb]{0.68,0.00,0.00}{#1}}
\newcommand{\FunctionTok}[1]{\textcolor[rgb]{0.28,0.35,0.67}{#1}}
\newcommand{\ImportTok}[1]{\textcolor[rgb]{0.00,0.46,0.62}{#1}}
\newcommand{\InformationTok}[1]{\textcolor[rgb]{0.37,0.37,0.37}{#1}}
\newcommand{\KeywordTok}[1]{\textcolor[rgb]{0.00,0.23,0.31}{\textbf{#1}}}
\newcommand{\NormalTok}[1]{\textcolor[rgb]{0.00,0.23,0.31}{#1}}
\newcommand{\OperatorTok}[1]{\textcolor[rgb]{0.37,0.37,0.37}{#1}}
\newcommand{\OtherTok}[1]{\textcolor[rgb]{0.00,0.23,0.31}{#1}}
\newcommand{\PreprocessorTok}[1]{\textcolor[rgb]{0.68,0.00,0.00}{#1}}
\newcommand{\RegionMarkerTok}[1]{\textcolor[rgb]{0.00,0.23,0.31}{#1}}
\newcommand{\SpecialCharTok}[1]{\textcolor[rgb]{0.37,0.37,0.37}{#1}}
\newcommand{\SpecialStringTok}[1]{\textcolor[rgb]{0.13,0.47,0.30}{#1}}
\newcommand{\StringTok}[1]{\textcolor[rgb]{0.13,0.47,0.30}{#1}}
\newcommand{\VariableTok}[1]{\textcolor[rgb]{0.07,0.07,0.07}{#1}}
\newcommand{\VerbatimStringTok}[1]{\textcolor[rgb]{0.13,0.47,0.30}{#1}}
\newcommand{\WarningTok}[1]{\textcolor[rgb]{0.37,0.37,0.37}{\textit{#1}}}

\providecommand{\tightlist}{%
  \setlength{\itemsep}{0pt}\setlength{\parskip}{0pt}}\usepackage{longtable,booktabs,array}
\usepackage{calc} % for calculating minipage widths
% Correct order of tables after \paragraph or \subparagraph
\usepackage{etoolbox}
\makeatletter
\patchcmd\longtable{\par}{\if@noskipsec\mbox{}\fi\par}{}{}
\makeatother
% Allow footnotes in longtable head/foot
\IfFileExists{footnotehyper.sty}{\usepackage{footnotehyper}}{\usepackage{footnote}}
\makesavenoteenv{longtable}
\usepackage{graphicx}
\makeatletter
\newsavebox\pandoc@box
\newcommand*\pandocbounded[1]{% scales image to fit in text height/width
  \sbox\pandoc@box{#1}%
  \Gscale@div\@tempa{\textheight}{\dimexpr\ht\pandoc@box+\dp\pandoc@box\relax}%
  \Gscale@div\@tempb{\linewidth}{\wd\pandoc@box}%
  \ifdim\@tempb\p@<\@tempa\p@\let\@tempa\@tempb\fi% select the smaller of both
  \ifdim\@tempa\p@<\p@\scalebox{\@tempa}{\usebox\pandoc@box}%
  \else\usebox{\pandoc@box}%
  \fi%
}
% Set default figure placement to htbp
\def\fps@figure{htbp}
\makeatother
% definitions for citeproc citations
\NewDocumentCommand\citeproctext{}{}
\NewDocumentCommand\citeproc{mm}{%
  \begingroup\def\citeproctext{#2}\cite{#1}\endgroup}
\makeatletter
 % allow citations to break across lines
 \let\@cite@ofmt\@firstofone
 % avoid brackets around text for \cite:
 \def\@biblabel#1{}
 \def\@cite#1#2{{#1\if@tempswa , #2\fi}}
\makeatother
\newlength{\cslhangindent}
\setlength{\cslhangindent}{1.5em}
\newlength{\csllabelwidth}
\setlength{\csllabelwidth}{3em}
\newenvironment{CSLReferences}[2] % #1 hanging-indent, #2 entry-spacing
 {\begin{list}{}{%
  \setlength{\itemindent}{0pt}
  \setlength{\leftmargin}{0pt}
  \setlength{\parsep}{0pt}
  % turn on hanging indent if param 1 is 1
  \ifodd #1
   \setlength{\leftmargin}{\cslhangindent}
   \setlength{\itemindent}{-1\cslhangindent}
  \fi
  % set entry spacing
  \setlength{\itemsep}{#2\baselineskip}}}
 {\end{list}}
\usepackage{calc}
\newcommand{\CSLBlock}[1]{\hfill\break\parbox[t]{\linewidth}{\strut\ignorespaces#1\strut}}
\newcommand{\CSLLeftMargin}[1]{\parbox[t]{\csllabelwidth}{\strut#1\strut}}
\newcommand{\CSLRightInline}[1]{\parbox[t]{\linewidth - \csllabelwidth}{\strut#1\strut}}
\newcommand{\CSLIndent}[1]{\hspace{\cslhangindent}#1}

\usepackage{fancyhdr}
\pagestyle{fancy}
\renewcommand{\footrulewidth}{0.4pt}
\setlength{\headsep}{50pt}
\fancyhead[L]{\includegraphics[width=1cm]{tidal_logo.png} \href{https://www.project-tidal.org/}{Project TIDAL}}
\fancyhead[R]{Module 1.3: Data Analytics Using R}
\fancyfoot[L]{\href{https://creativecommons.org/licenses/by-nc-nd/4.0/}{\includegraphics[width=2cm]{cc_by_nc_nd.png}}}
\fancyfoot[C]{\url{emorytidal.netlify.app}}
\fancyfoot[R]{\thepage}
\usepackage{longtable}
\KOMAoption{captions}{tableheading}
\makeatletter
\@ifpackageloaded{tcolorbox}{}{\usepackage[skins,breakable]{tcolorbox}}
\@ifpackageloaded{fontawesome5}{}{\usepackage{fontawesome5}}
\definecolor{quarto-callout-color}{HTML}{909090}
\definecolor{quarto-callout-note-color}{HTML}{0758E5}
\definecolor{quarto-callout-important-color}{HTML}{CC1914}
\definecolor{quarto-callout-warning-color}{HTML}{EB9113}
\definecolor{quarto-callout-tip-color}{HTML}{00A047}
\definecolor{quarto-callout-caution-color}{HTML}{FC5300}
\definecolor{quarto-callout-color-frame}{HTML}{acacac}
\definecolor{quarto-callout-note-color-frame}{HTML}{4582ec}
\definecolor{quarto-callout-important-color-frame}{HTML}{d9534f}
\definecolor{quarto-callout-warning-color-frame}{HTML}{f0ad4e}
\definecolor{quarto-callout-tip-color-frame}{HTML}{02b875}
\definecolor{quarto-callout-caution-color-frame}{HTML}{fd7e14}
\makeatother
\makeatletter
\@ifpackageloaded{caption}{}{\usepackage{caption}}
\AtBeginDocument{%
\ifdefined\contentsname
  \renewcommand*\contentsname{Table of contents}
\else
  \newcommand\contentsname{Table of contents}
\fi
\ifdefined\listfigurename
  \renewcommand*\listfigurename{List of Figures}
\else
  \newcommand\listfigurename{List of Figures}
\fi
\ifdefined\listtablename
  \renewcommand*\listtablename{List of Tables}
\else
  \newcommand\listtablename{List of Tables}
\fi
\ifdefined\figurename
  \renewcommand*\figurename{Figure}
\else
  \newcommand\figurename{Figure}
\fi
\ifdefined\tablename
  \renewcommand*\tablename{Table}
\else
  \newcommand\tablename{Table}
\fi
}
\@ifpackageloaded{float}{}{\usepackage{float}}
\floatstyle{ruled}
\@ifundefined{c@chapter}{\newfloat{codelisting}{h}{lop}}{\newfloat{codelisting}{h}{lop}[chapter]}
\floatname{codelisting}{Listing}
\newcommand*\listoflistings{\listof{codelisting}{List of Listings}}
\makeatother
\makeatletter
\makeatother
\makeatletter
\@ifpackageloaded{caption}{}{\usepackage{caption}}
\@ifpackageloaded{subcaption}{}{\usepackage{subcaption}}
\makeatother

\usepackage{bookmark}

\IfFileExists{xurl.sty}{\usepackage{xurl}}{} % add URL line breaks if available
\urlstyle{same} % disable monospaced font for URLs
\hypersetup{
  pdftitle={1.3.3: Data Visualization},
  colorlinks=true,
  linkcolor={blue},
  filecolor={Maroon},
  citecolor={Blue},
  urlcolor={Blue},
  pdfcreator={LaTeX via pandoc}}


\title{1.3.3: Data Visualization}
\usepackage{etoolbox}
\makeatletter
\providecommand{\subtitle}[1]{% add subtitle to \maketitle
  \apptocmd{\@title}{\par {\large #1 \par}}{}{}
}
\makeatother
\subtitle{(Asynchronous-Online)}
\author{}
\date{}

\begin{document}
\maketitle


\thispagestyle{fancy}

\subsection{Session Objectives}\label{session-objectives}

\begin{enumerate}
\def\labelenumi{\arabic{enumi}.}
\tightlist
\item
  To visualize data using different R packages.
\end{enumerate}

This lesson module will include:

\begin{itemize}
\tightlist
\item
  Introductions to \texttt{ggplot2} and other relevant R packages for
  graphics.
\item
  Visualizing one, two, and more variables at a time.
\item
  Summary Tables with Graphics
\item
  Lists of other resources: books, blogs, websites, etc.
\end{itemize}

\begin{center}\rule{0.5\linewidth}{0.5pt}\end{center}

\subsection{0. Prework - Before You
Begin}\label{prework---before-you-begin}

\subsubsection{A. Install packages}\label{a.-install-packages}

If you do not have them already, install the following packages from
CRAN:

\begin{itemize}
\tightlist
\item
  \href{https://cloud.r-project.org/web/packages/ggplot2/}{\texttt{ggplot2}}
\item
  \href{https://cloud.r-project.org/web/packages/dplyr/}{\texttt{dplyr}}
\item
  \href{https://cran.r-project.org/web/packages/patchwork/}{\texttt{patchwork}}
\item
  \href{https://cran.r-project.org/web/packages/ggpubr/}{\texttt{ggpubr}}
\item
  \href{https://cran.r-project.org/web/packages/GGally/}{\texttt{GGally}}
\item
  \href{https://cran.r-project.org/web/packages/vcd/}{\texttt{vcd}}
\item
  Optional
  \href{https://cran.r-project.org/web/packages/gapminder/}{\texttt{gapminder}}
\item
  Optional
  \href{https://cran.r-project.org/web/packages/gganimate/}{\texttt{gganimate}}
\item
  Optional
  \href{https://cran.r-project.org/web/packages/plotly/}{\texttt{plotly}}
\item
  Optional
  \href{https://cran.r-project.org/web/packages/gt/}{\texttt{gt}}
\item
  Optional
  \href{https://cran.r-project.org/web/packages/gtExtras/}{\texttt{gtExtras}}
\end{itemize}

\subsubsection{B. Open/create your RStudio
project}\label{b.-opencreate-your-rstudio-project}

Let's start with the \texttt{myfirstRproject} RStudio project you
created in
\href{module132_DataWrangling.html\#begin-with-a-new-rstudio-project}{Module
1.3.2 - part 1}. If you have not yet created this
\texttt{myfirstRproject} RStudio project, go ahead and create a new
RStudio Project for this lesson. \emph{Feel free to name your project
whatever you want, it does not need to be named
\texttt{myfirstRproject}.}

\subsubsection{C. Create a new R script and load data into your
computing
session}\label{c.-create-a-new-r-script-and-load-data-into-your-computing-session}

At the end of
\href{module132_DataWrangling.html\#save-mydata-as-.rdata-native-r-binary-format}{Module
1.3.2 - part 6} you saved the \texttt{mydata} dataset in the
\texttt{mydata.RData} R binary format.

\begin{enumerate}
\def\labelenumi{\arabic{enumi}.}
\item
  Go ahead and create a new R script (\texttt{*.R}) for this computing
  session. \emph{We did this already in
  \href{module131_IntroRRStudio.html\#create-your-first-r-script}{Module
  1.3.1 - part 3} - refer to this section to remember how to create a
  new R script.}
\item
  Put this code into your new R script (\texttt{*.R}) to load
  \texttt{mydata.RData} into your current computing session.
\end{enumerate}

\begin{Shaded}
\begin{Highlighting}[]
\CommentTok{\# load mydata}
\FunctionTok{load}\NormalTok{(}\AttributeTok{file =} \StringTok{"mydata.RData"}\NormalTok{)}
\end{Highlighting}
\end{Shaded}

\begin{tcolorbox}[enhanced jigsaw, arc=.35mm, colback=white, rightrule=.15mm, toprule=.15mm, colframe=quarto-callout-important-color-frame, opacityback=0, breakable, titlerule=0mm, left=2mm, title=\textcolor{quarto-callout-important-color}{\faExclamation}\hspace{0.5em}{Data must/should be in your RStudio project}, toptitle=1mm, opacitybacktitle=0.6, bottomtitle=1mm, leftrule=.75mm, bottomrule=.15mm, coltitle=black, colbacktitle=quarto-callout-important-color!10!white]

\textbf{REMEMBER} R/RStudio automatically looks in your current RStudio
project folder for all files for your current computing session. So,
make sure the \texttt{mydata.RData} file is in your current RStudio
project \texttt{myfirstRproject} folder on your computer.

\medskip

For a more detailed overview of RStudio projects:

\begin{itemize}
\tightlist
\item
  read
  \href{https://epirhandbook.com/en/new_pages/r_projects.html}{``Chapter
  6: R projects'' in the \emph{The Epidemiologist R Handbook}} and
\item
  refer to
  \href{https://epirhandbook.com/en/new_pages/directories.html}{``Chapter
  45 Directory interactions'' in the \emph{The Epidemiologist R
  Handbook}}.
\end{itemize}

\end{tcolorbox}

\subsubsection{D. Get Inspired!}\label{d.-get-inspired}

\begin{itemize}
\tightlist
\item
  Get Inspired at \href{https://r-graph-gallery.com/}{The R Graph
  Gallery}
\item
  Also see the
  \href{https://r-graph-gallery.com/best-r-chart-examples}{Top Curated R
  Graphs}
\item
  Also see \href{additionalResources.html\#r-graphics}{Additional
  Resources - R Graphics}
\end{itemize}

\begin{center}\rule{0.5\linewidth}{0.5pt}\end{center}

\newpage

\subsection{1. Base R graphical
functions}\label{base-r-graphical-functions}

The base R \texttt{graphics} package is very powerful on its own. As you
saw in \href{module131_IntroRRStudio.html}{1.3.1: Introduction to R and
R Studio}, we can make a simple 2-dimensional scatterplot with the
\texttt{plot()} function.

\subsubsection{Base R - Scatterplot}\label{base-r---scatterplot}

For example, let's make a plot of \texttt{Height} on the X-axis
(horizontal) and \texttt{WeightPRE} on the Y-axis (vertical) from the
\texttt{mydata} dataset. Since we are using base R function, we have to
use the \texttt{\$}selector to identify the variables we want inside the
\texttt{mydata} dataset.

Learn more about the \texttt{plot()} function and arguments by running
\texttt{help(plot,\ package\ =\ "graphics")}.

\begin{Shaded}
\begin{Highlighting}[]
\FunctionTok{plot}\NormalTok{(}\AttributeTok{x =}\NormalTok{ mydata}\SpecialCharTok{$}\NormalTok{Height,}
     \AttributeTok{y =}\NormalTok{ mydata}\SpecialCharTok{$}\NormalTok{WeightPRE)}
\end{Highlighting}
\end{Shaded}

\pandocbounded{\includegraphics[keepaspectratio]{module133_DataVis_files/figure-pdf/unnamed-chunk-2-1.pdf}}

The plot does look a little odd - this is due to some data errors in the
\texttt{mydata} dataset. We will fix these below. But for now, you can
``see'' that these data may have some issues that need to be addressed.
For example:

\begin{itemize}
\tightlist
\item
  There are 2 people with heights \textless{} 5 feet tall which may be
  suspect
\item
  There are 2 people with a weight \textless{} 100 pounds which may be
  data entry errors or incorrect units
\end{itemize}

\newpage

For now, let's add some additional graphical elements:

\begin{itemize}
\tightlist
\item
  a better label for the x-axis
\item
  a better label for the y-axis
\item
  a title for the graph
\item
  a subtitle for the graph
\end{itemize}

\begin{Shaded}
\begin{Highlighting}[]
\FunctionTok{plot}\NormalTok{(}\AttributeTok{x =}\NormalTok{ mydata}\SpecialCharTok{$}\NormalTok{Height,}
     \AttributeTok{y =}\NormalTok{ mydata}\SpecialCharTok{$}\NormalTok{WeightPRE,}
     \AttributeTok{xlab =} \StringTok{"Height (in decimal inches)"}\NormalTok{,}
     \AttributeTok{ylab =} \StringTok{"Weight (in pounds) {-} before intervention"}\NormalTok{,}
     \AttributeTok{main =} \StringTok{"Weight by Height in the Mydata Project"}\NormalTok{,}
     \AttributeTok{sub =} \StringTok{"Hypothetical Madeup mydata Dataset"}\NormalTok{)}
\end{Highlighting}
\end{Shaded}

\pandocbounded{\includegraphics[keepaspectratio]{module133_DataVis_files/figure-pdf/unnamed-chunk-3-1.pdf}}

\newpage

And we could also add color and change the shapes - for example, let's
color and shape the points by \texttt{GenderCoded}, the numeric coding
for gender where 1=Male, 2=Female.

And we can add a legend inside the plot as well.

\begin{tcolorbox}[enhanced jigsaw, arc=.35mm, colback=white, rightrule=.15mm, toprule=.15mm, colframe=quarto-callout-note-color-frame, opacityback=0, breakable, titlerule=0mm, left=2mm, title=\textcolor{quarto-callout-note-color}{\faInfo}\hspace{0.5em}{Plot code inspiration}, toptitle=1mm, opacitybacktitle=0.6, bottomtitle=1mm, leftrule=.75mm, bottomrule=.15mm, coltitle=black, colbacktitle=quarto-callout-note-color!10!white]

I pulled this code together from code examples at:

\begin{itemize}
\tightlist
\item
  \href{https://stackoverflow.com/questions/12919816/plotting-in-different-shapes-using-pch-argument}{Stackoverflow
  post on using \texttt{pch}}
\item
  \href{https://www.sthda.com/english/wiki/r-plot-pch-symbols-the-different-point-shapes-available-in-r}{STHDA
  post on point shapes}
\item
  \href{https://www.sthda.com/english/wiki/add-legends-to-plots-in-r-software-the-easiest-way}{STDHA
  post on base R legends}
\item
  \href{https://r-graph-gallery.com/119-add-a-legend-to-a-plot.html}{R-Graph
  Galler post on base R legends}
\end{itemize}

\end{tcolorbox}

\begin{Shaded}
\begin{Highlighting}[]
\FunctionTok{plot}\NormalTok{(}\AttributeTok{x =}\NormalTok{ mydata}\SpecialCharTok{$}\NormalTok{Height,}
     \AttributeTok{y =}\NormalTok{ mydata}\SpecialCharTok{$}\NormalTok{WeightPRE,}
     \AttributeTok{col =} \FunctionTok{c}\NormalTok{(}\StringTok{"blue"}\NormalTok{, }\StringTok{"green"}\NormalTok{)[mydata}\SpecialCharTok{$}\NormalTok{GenderCoded],}
     \AttributeTok{pch =} \FunctionTok{c}\NormalTok{(}\DecValTok{15}\NormalTok{, }\DecValTok{19}\NormalTok{)[mydata}\SpecialCharTok{$}\NormalTok{GenderCoded],}
     \AttributeTok{xlab =} \StringTok{"Height (in decimal inches)"}\NormalTok{,}
     \AttributeTok{ylab =} \StringTok{"Weight (in pounds) {-} before intervention"}\NormalTok{,}
     \AttributeTok{main =} \StringTok{"Weight by Height in the Mydata Project"}\NormalTok{,}
     \AttributeTok{sub =} \StringTok{"Hypothetical Madeup mydata Dataset"}\NormalTok{)}
\FunctionTok{legend}\NormalTok{(}\DecValTok{3}\NormalTok{, }\DecValTok{250}\NormalTok{, }\AttributeTok{legend=}\FunctionTok{c}\NormalTok{(}\StringTok{"Male"}\NormalTok{, }\StringTok{"Female"}\NormalTok{),}
       \AttributeTok{col=}\FunctionTok{c}\NormalTok{(}\StringTok{"blue"}\NormalTok{, }\StringTok{"green"}\NormalTok{), }\AttributeTok{pch =} \FunctionTok{c}\NormalTok{(}\DecValTok{15}\NormalTok{, }\DecValTok{19}\NormalTok{), }\AttributeTok{cex=}\FloatTok{0.8}\NormalTok{)}
\end{Highlighting}
\end{Shaded}

\pandocbounded{\includegraphics[keepaspectratio]{module133_DataVis_files/figure-pdf/unnamed-chunk-4-1.pdf}}

The \href{https://www.sthda.com/english/wiki/r-base-graphs}{STHDA
website on ``R Base Graphs''} has a nice walk through of using the base
R \texttt{graphics} package to make really nice plots.

\newpage

\subsubsection{Base R - Histogram}\label{base-r---histogram}

\paragraph{\texorpdfstring{\textbf{Basic
Histogram}}{Basic Histogram}}\label{basic-histogram}

As we noted above, let's take a look at the distribution of the heights
in the \texttt{mydata} dataset. There is a specific \texttt{hist()}
function in the \texttt{graphics} package for making histograms, learn
more by running \texttt{help(hist,\ package\ =\ "graphics")}.

Notice that we can use some of the same arguments as we did above for
\texttt{plot()}.

\begin{Shaded}
\begin{Highlighting}[]
\FunctionTok{hist}\NormalTok{(mydata}\SpecialCharTok{$}\NormalTok{Height,}
     \AttributeTok{xlab =} \StringTok{"Height (in decimal inches)"}\NormalTok{,}
     \AttributeTok{col =} \StringTok{"lightblue"}\NormalTok{,}
     \AttributeTok{border =} \StringTok{"black"}\NormalTok{,}
     \AttributeTok{main =} \StringTok{"Histogram of Heights"}\NormalTok{,}
     \AttributeTok{sub =} \StringTok{"Hypothetical Madeup mydata Dataset"}\NormalTok{)}
\end{Highlighting}
\end{Shaded}

\pandocbounded{\includegraphics[keepaspectratio]{module133_DataVis_files/figure-pdf/unnamed-chunk-5-1.pdf}}

\begin{tcolorbox}[enhanced jigsaw, arc=.35mm, colback=white, rightrule=.15mm, toprule=.15mm, colframe=quarto-callout-tip-color-frame, opacityback=0, breakable, titlerule=0mm, left=2mm, title=\textcolor{quarto-callout-tip-color}{\faLightbulb}\hspace{0.5em}{Colors available}, toptitle=1mm, opacitybacktitle=0.6, bottomtitle=1mm, leftrule=.75mm, bottomrule=.15mm, coltitle=black, colbacktitle=quarto-callout-tip-color!10!white]

There are 657 names of colors immediately available to you from the
built-in \texttt{grDevices} Base R package which works in conjunction
with \texttt{graphics}. You can view the names of all of these colors by
running \texttt{colors()}. You can also learn more at:

\begin{itemize}
\tightlist
\item
  \url{https://www.sthda.com/english/wiki/colors-in-r\#google_vignette}
\item
  \url{https://r-graph-gallery.com/42-colors-names.html}
\item
  \url{https://r-graph-gallery.com/ggplot2-color.html} - which explains
  how colors can be specified using the built-in color names, but can
  also be specified using RGB (red, green, blue) indexes or even
  Hexcodes for which there are many online tools like
  \url{https://htmlcolorcodes.com/}.
\end{itemize}

\end{tcolorbox}

\begin{Shaded}
\begin{Highlighting}[]
\CommentTok{\# code not run here {-} do in your session}
\CommentTok{\# list built{-}in colors}
\FunctionTok{colors}\NormalTok{()}
\end{Highlighting}
\end{Shaded}

\newpage

\paragraph{\texorpdfstring{\textbf{Histogram with Overlaid Density
Curve}}{Histogram with Overlaid Density Curve}}\label{histogram-with-overlaid-density-curve}

Statisticians often like seeing a histogram \emph{(for the frequencies
or probability of each value for the variable in the dataset)} with an
overlaid density curve \emph{(which is ``smoothed'' line for these
probabilities)}. Statistical software like SAS and SPSS make this really
easy. However, in R, we need to think through the process to get this to
work.

\begin{itemize}
\tightlist
\item
  First, we need to make the histogram using probabilities for the
  ``bars'' in the histogram instead of frequency counts.
\item
  Second, we need to add a density line curve over the histogram
  ``bars''.
\end{itemize}

See these online examples:

\begin{itemize}
\tightlist
\item
  \url{https://r-charts.com/distribution/histogram-curves/}
\item
  \url{https://www.datacamp.com/doc/r/histograms-and-density}
\item
  \url{https://www.r-bloggers.com/2012/09/histogram-density-plot-combo-in-r/}
\end{itemize}

\begin{Shaded}
\begin{Highlighting}[]
\CommentTok{\# make histogram as we did above}
\CommentTok{\# add freq = FALSE}
\FunctionTok{hist}\NormalTok{(mydata}\SpecialCharTok{$}\NormalTok{Height,}
     \AttributeTok{freq =} \ConstantTok{FALSE}\NormalTok{,}
     \AttributeTok{xlab =} \StringTok{"Height (in decimal inches)"}\NormalTok{,}
     \AttributeTok{col =} \StringTok{"lightblue"}\NormalTok{,}
     \AttributeTok{border =} \StringTok{"black"}\NormalTok{,}
     \AttributeTok{main =} \StringTok{"Histogram of Heights"}\NormalTok{,}
     \AttributeTok{sub =} \StringTok{"Hypothetical Madeup mydata Dataset"}\NormalTok{)}

\CommentTok{\# add density curve line}
\CommentTok{\# add na.rm=TRUE to remove }
\CommentTok{\# the missing values in Height}
\FunctionTok{lines}\NormalTok{(}\FunctionTok{density}\NormalTok{(mydata}\SpecialCharTok{$}\NormalTok{Height, }\AttributeTok{na.rm=}\ConstantTok{TRUE}\NormalTok{),}
      \AttributeTok{col =} \StringTok{"black"}\NormalTok{)}
\end{Highlighting}
\end{Shaded}

\pandocbounded{\includegraphics[keepaspectratio]{module133_DataVis_files/figure-pdf/unnamed-chunk-7-1.pdf}}

\newpage

\paragraph{\texorpdfstring{\textbf{Fix the
Heights}}{Fix the Heights}}\label{fix-the-heights}

So as you can see in the histogram and in the scatterplot figures above
for the \texttt{Height} variable, there are 2 people with heights under
4 feet tall.

\begin{Shaded}
\begin{Highlighting}[]
\CommentTok{\# use dplyr::arrange()}
\FunctionTok{library}\NormalTok{(dplyr)}

\NormalTok{mydata }\SpecialCharTok{\%\textgreater{}\%}
  \FunctionTok{select}\NormalTok{(SubjectID, Height) }\SpecialCharTok{\%\textgreater{}\%}
  \FunctionTok{arrange}\NormalTok{(Height) }\SpecialCharTok{\%\textgreater{}\%}
  \FunctionTok{head}\NormalTok{()}
\end{Highlighting}
\end{Shaded}

\begin{verbatim}
# A tibble: 6 x 2
  SubjectID Height
      <dbl>  <dbl>
1        28    2.6
2         8    3.3
3         9    5.1
4         6    5.2
5         2    5.4
6        12    5.5
\end{verbatim}

Let's look at these values:

\begin{itemize}
\tightlist
\item
  \texttt{SubjectID} number 28 has a \texttt{Height} of 2.6 feet tall

  \begin{itemize}
  \tightlist
  \item
    If this wasn't a made-up dataset, we could ask the original data
    collectors to see if there is a way to check this value in their
    records or possibly to re-measure this individual.
  \item
    For now, let's assume this was a simple typo where the 2 numbers
    were transposed where this individual should be 6.2 feet tall.
  \end{itemize}
\item
  \texttt{SubjectID} number 8 has a \texttt{Height} of 3.3 feet tall

  \begin{itemize}
  \tightlist
  \item
    Unfortunately, this is probably not a simple typo. Without further
    details, we should maybe set this to missing as an invalidated data
    point.
  \item
    As a side-note, I actually ran into this problem in a study where
    one of the participants was a paraplegic. So, this could be a
    legitimate height. But when computing BMI, adjustments need to be
    made or alternative body metrics are needed.
  \item
    For now, we will set this to missing, \texttt{NA\_real\_} which is
    missing for ``real'' numeric variables.
  \end{itemize}
\end{itemize}

\begin{Shaded}
\begin{Highlighting}[]
\CommentTok{\# make a copy of the dataset}
\NormalTok{mydata\_corrected }\OtherTok{\textless{}{-}}\NormalTok{ mydata}

\CommentTok{\# compute a new corrected height}
\CommentTok{\# fix heights for these 2 IDs}
\NormalTok{mydata\_corrected }\OtherTok{\textless{}{-}} 
\NormalTok{  mydata\_corrected }\SpecialCharTok{\%\textgreater{}\%}
  \FunctionTok{mutate}\NormalTok{(}\AttributeTok{Height\_corrected =} \FunctionTok{case\_when}\NormalTok{(}
\NormalTok{    (SubjectID }\SpecialCharTok{==} \DecValTok{28}\NormalTok{) }\SpecialCharTok{\textasciitilde{}} \FloatTok{6.2}\NormalTok{,}
\NormalTok{    (SubjectID }\SpecialCharTok{==} \DecValTok{8}\NormalTok{) }\SpecialCharTok{\textasciitilde{}} \ConstantTok{NA\_real\_}\NormalTok{,}
    \AttributeTok{.default =}\NormalTok{ Height}
\NormalTok{  ))}
\end{Highlighting}
\end{Shaded}

\newpage

Remake the histogram with the corrected heights.

\begin{Shaded}
\begin{Highlighting}[]
\CommentTok{\# make histogram as we did above}
\CommentTok{\# add freq = FALSE}
\FunctionTok{hist}\NormalTok{(mydata\_corrected}\SpecialCharTok{$}\NormalTok{Height\_corrected,}
     \AttributeTok{freq =} \ConstantTok{FALSE}\NormalTok{,}
     \AttributeTok{xlab =} \StringTok{"Height (in decimal inches)"}\NormalTok{,}
     \AttributeTok{col =} \StringTok{"lightblue"}\NormalTok{,}
     \AttributeTok{border =} \StringTok{"black"}\NormalTok{,}
     \AttributeTok{main =} \StringTok{"Histogram of Heights"}\NormalTok{,}
     \AttributeTok{sub =} \StringTok{"Hypothetical Madeup mydata Dataset"}\NormalTok{)}

\CommentTok{\# add density curve line}
\CommentTok{\# add na.rm=TRUE to remove }
\CommentTok{\# the missing values in Height}
\FunctionTok{lines}\NormalTok{(}\FunctionTok{density}\NormalTok{(mydata\_corrected}\SpecialCharTok{$}\NormalTok{Height\_corrected, }\AttributeTok{na.rm=}\ConstantTok{TRUE}\NormalTok{),}
      \AttributeTok{col =} \StringTok{"black"}\NormalTok{)}
\end{Highlighting}
\end{Shaded}

\pandocbounded{\includegraphics[keepaspectratio]{module133_DataVis_files/figure-pdf/unnamed-chunk-10-1.pdf}}

\newpage

\subsubsection{Base R - Barchart}\label{base-r---barchart}

Let's make a bar chart for the frequencies for the 3 SES categories:

\begin{itemize}
\tightlist
\item
  fill the bars with a yellow color specified by the HEX code
  \href{https://www.color-hex.com/color/f7f445}{\texttt{\#f7f445}}
\item
  set the border color as \texttt{darkgreen} and make the border line
  thicker by updating the \texttt{lwd}, see
  \href{https://stackoverflow.com/questions/8795862/width-of-edge-of-the-bars-in-r-plots}{Stack
  Overflow Post on bar width}.
\end{itemize}

\begin{Shaded}
\begin{Highlighting}[]
\CommentTok{\# get table of frequencies for each category}
\NormalTok{tab1 }\OtherTok{\textless{}{-}} \FunctionTok{table}\NormalTok{(mydata\_corrected}\SpecialCharTok{$}\NormalTok{SES.f)}

\NormalTok{opar }\OtherTok{\textless{}{-}} \FunctionTok{par}\NormalTok{() }\CommentTok{\# save current plotting parameters}
\FunctionTok{par}\NormalTok{(}\AttributeTok{lwd =} \DecValTok{3}\NormalTok{) }\CommentTok{\# change border linewidth}

\CommentTok{\# make plot of the frequencies for }
\CommentTok{\# each category}
\FunctionTok{barplot}\NormalTok{(tab1,}
        \AttributeTok{xlab =} \StringTok{"SES Categories"}\NormalTok{,}
        \AttributeTok{ylab =} \StringTok{"Frequencies"}\NormalTok{,}
        \AttributeTok{col =} \StringTok{"\#f7f445"}\NormalTok{,}
        \AttributeTok{border =} \StringTok{"darkgreen"}\NormalTok{,}
        \AttributeTok{main =} \StringTok{"Socio Economic Status Categories"}\NormalTok{,}
        \AttributeTok{sub =} \StringTok{"Hypothetical Madeup mydata Dataset"}\NormalTok{)}
\end{Highlighting}
\end{Shaded}

\pandocbounded{\includegraphics[keepaspectratio]{module133_DataVis_files/figure-pdf/unnamed-chunk-11-1.pdf}}

\begin{Shaded}
\begin{Highlighting}[]
\FunctionTok{par}\NormalTok{(opar) }\CommentTok{\# reset plotting parameters to defaults}
\end{Highlighting}
\end{Shaded}

\newpage

\subsubsection{Base R - Boxplot}\label{base-r---boxplot}

Make side-by-side boxplots of the heights by gender.

\begin{Shaded}
\begin{Highlighting}[]
\FunctionTok{boxplot}\NormalTok{(Height\_corrected }\SpecialCharTok{\textasciitilde{}}\NormalTok{ GenderCoded.f,}
        \AttributeTok{data =}\NormalTok{ mydata\_corrected,}
        \AttributeTok{xlab =} \StringTok{"Gender"}\NormalTok{,}
        \AttributeTok{ylab =} \StringTok{"Height (in decimal feet)"}\NormalTok{,}
        \AttributeTok{col =} \StringTok{"\#f58ef1"}\NormalTok{,}
        \AttributeTok{border =} \StringTok{"darkmagenta"}\NormalTok{,}
        \AttributeTok{main =} \StringTok{"Height by Gender"}\NormalTok{,}
        \AttributeTok{sub =} \StringTok{"Hypothetical Madeup mydata Dataset"}\NormalTok{)}
\end{Highlighting}
\end{Shaded}

\pandocbounded{\includegraphics[keepaspectratio]{module133_DataVis_files/figure-pdf/unnamed-chunk-12-1.pdf}}

\begin{center}\rule{0.5\linewidth}{0.5pt}\end{center}

\newpage

\subsection{\texorpdfstring{2. The \texttt{ggplot2}
package}{2. The ggplot2 package}}\label{the-ggplot2-package}

The \texttt{ggplot2} package name starts with \texttt{gg} which stands
for the ``grammar of graphics'' which is explained in the
\href{https://ggplot2-book.org/introduction\#what-is-the-grammar-of-graphics}{``ggplot2:
Elegant Graphics for Data Analysis (3e)'' Book}.

\begin{tcolorbox}[enhanced jigsaw, arc=.35mm, colback=white, rightrule=.15mm, toprule=.15mm, colframe=quarto-callout-note-color-frame, opacityback=0, breakable, titlerule=0mm, left=2mm, title=\textcolor{quarto-callout-note-color}{\faInfo}\hspace{0.5em}{Why is the package \texttt{ggplot2} and not \texttt{ggplot}?}, toptitle=1mm, opacitybacktitle=0.6, bottomtitle=1mm, leftrule=.75mm, bottomrule=.15mm, coltitle=black, colbacktitle=quarto-callout-note-color!10!white]

Many people often ask Hadley Wickham (the developer of \texttt{ggplot2})
what happened to the first \texttt{ggplot}? Technically, there was a
\texttt{ggplot} package and you can still view the
\href{https://cran.r-project.org/src/contrib/Archive/ggplot/}{\texttt{ggplot}
archived package versions on CRAN} which date back to 2006 with the last
version posted in 2008. However, in 2007, Hadley redesigned the package
and published the first version of
\href{https://cran.r-project.org/src/contrib/Archive/ggplot2/}{\texttt{ggplot2}
\emph{(version 0.5.1)} was posted on CRAN}. So, \texttt{ggplot2} is the
package that has stayed in production and actively maintained for nearly
20 years!!

\end{tcolorbox}

Given that \texttt{ggplot2} has been actively maintained for nearly 20
years, it has become \emph{almost} the defacto graphical standard for R
graphics. If you take a look at the
\href{https://cran.r-project.org/web/packages/available_packages_by_name.html\#available-packages-G}{list
of packages on CRAN that start with the letter ``G''}, as of this
morning \texttt{01/28/2025\ at\ 8:23\ am\ EST,\ USA}, there are 230
packages that start with \texttt{gg} - nearly all of these are
compatible packages that extend the functionality or work in concert
with the \texttt{ggplot2} package. There are also currently 14 packages
on the
\href{https://www.bioconductor.org/packages/release/BiocViews.html\#___Software}{Bioconductor
repository} that start with \texttt{gg}.

Let's make plots similar to the ones above but now using
\texttt{ggplot2}. When making a \texttt{ggplot2} plot, we build the
plots using layers that get added to the previous layers.

\newpage

\subsubsection{\texorpdfstring{\texttt{ggplot2} -
Scatterplot}{ggplot2 - Scatterplot}}\label{ggplot2---scatterplot}

Here are the steps to building a scatterplot.

\begin{enumerate}
\def\labelenumi{\arabic{enumi}.}
\tightlist
\item
  First, load the \texttt{ggplot2} package, designate the dataset and
  variables (aesthetics) to be included. This creates a plot space with
  nothing in it - we will add data in the next steps below.
\end{enumerate}

\begin{Shaded}
\begin{Highlighting}[]
\CommentTok{\#load ggplot2}
\FunctionTok{library}\NormalTok{(ggplot2)}

\CommentTok{\# create the plot space}
\FunctionTok{ggplot}\NormalTok{(}\AttributeTok{data =}\NormalTok{ mydata\_corrected,}
       \FunctionTok{aes}\NormalTok{(}\AttributeTok{x =}\NormalTok{ WeightPRE,}
           \AttributeTok{y =}\NormalTok{ WeightPOST))}
\end{Highlighting}
\end{Shaded}

\pandocbounded{\includegraphics[keepaspectratio]{module133_DataVis_files/figure-pdf/unnamed-chunk-13-1.pdf}}

\newpage

\begin{enumerate}
\def\labelenumi{\arabic{enumi}.}
\setcounter{enumi}{1}
\tightlist
\item
  Next add \texttt{+} a ``geometric object'' or ``geom'' to show the
  data as points.
\end{enumerate}

\begin{Shaded}
\begin{Highlighting}[]
\FunctionTok{ggplot}\NormalTok{(}\AttributeTok{data =}\NormalTok{ mydata\_corrected,}
       \FunctionTok{aes}\NormalTok{(}\AttributeTok{x =}\NormalTok{ WeightPRE,}
           \AttributeTok{y =}\NormalTok{ WeightPOST)) }\SpecialCharTok{+}
  \FunctionTok{geom\_point}\NormalTok{()}
\end{Highlighting}
\end{Shaded}

\pandocbounded{\includegraphics[keepaspectratio]{module133_DataVis_files/figure-pdf/unnamed-chunk-14-1.pdf}}

\newpage

\begin{enumerate}
\def\labelenumi{\arabic{enumi}.}
\setcounter{enumi}{2}
\tightlist
\item
  We can add color by \texttt{GenderCoded.f}
\end{enumerate}

\begin{tcolorbox}[enhanced jigsaw, arc=.35mm, colback=white, rightrule=.15mm, toprule=.15mm, colframe=quarto-callout-note-color-frame, opacityback=0, breakable, titlerule=0mm, left=2mm, title=\textcolor{quarto-callout-note-color}{\faInfo}\hspace{0.5em}{Automatic Legend}, toptitle=1mm, opacitybacktitle=0.6, bottomtitle=1mm, leftrule=.75mm, bottomrule=.15mm, coltitle=black, colbacktitle=quarto-callout-note-color!10!white]

Notice that by adding \texttt{color\ =\ GenderCoded.f} inside the
\texttt{aes()} aesthetic that a legend for the coloring of the points is
automatically added to the plot. This can be disabled if you wish. Learn
more about colors and legends in the
\href{https://ggplot2-book.org/scales-colour}{\texttt{ggplot2} book -
Chapter 11}.

\end{tcolorbox}

\begin{Shaded}
\begin{Highlighting}[]
\FunctionTok{ggplot}\NormalTok{(}\AttributeTok{data =}\NormalTok{ mydata\_corrected,}
       \FunctionTok{aes}\NormalTok{(}\AttributeTok{x =}\NormalTok{ WeightPRE,}
           \AttributeTok{y =}\NormalTok{ WeightPOST,}
           \AttributeTok{color =}\NormalTok{ GenderCoded.f)) }\SpecialCharTok{+}
  \FunctionTok{geom\_point}\NormalTok{()}
\end{Highlighting}
\end{Shaded}

\pandocbounded{\includegraphics[keepaspectratio]{module133_DataVis_files/figure-pdf/unnamed-chunk-15-1.pdf}}

\newpage

\begin{enumerate}
\def\labelenumi{\arabic{enumi}.}
\setcounter{enumi}{3}
\tightlist
\item
  We can also add labels, a title and better legend title
\end{enumerate}

\begin{Shaded}
\begin{Highlighting}[]
\FunctionTok{ggplot}\NormalTok{(}\AttributeTok{data =}\NormalTok{ mydata\_corrected,}
       \FunctionTok{aes}\NormalTok{(}\AttributeTok{x =}\NormalTok{ WeightPRE,}
           \AttributeTok{y =}\NormalTok{ WeightPOST,}
           \AttributeTok{color =}\NormalTok{ GenderCoded.f)) }\SpecialCharTok{+}
  \FunctionTok{geom\_point}\NormalTok{() }\SpecialCharTok{+}
  \FunctionTok{xlab}\NormalTok{(}\StringTok{"Weight (in pounds) before program"}\NormalTok{) }\SpecialCharTok{+}
  \FunctionTok{ylab}\NormalTok{(}\StringTok{"Weight (in pounds) after program"}\NormalTok{) }\SpecialCharTok{+}
  \FunctionTok{labs}\NormalTok{(}
    \AttributeTok{title =} \StringTok{"Weights (in pounds) before and after"}\NormalTok{,}
    \AttributeTok{subtitle =} \StringTok{"Hypothetical Madeup mydata Dataset"}\NormalTok{,}
    \AttributeTok{color =} \StringTok{"Gender"}
\NormalTok{  ) }
\end{Highlighting}
\end{Shaded}

\pandocbounded{\includegraphics[keepaspectratio]{module133_DataVis_files/figure-pdf/unnamed-chunk-16-1.pdf}}

Notice that there are 4 weights that seem off. Also notice that the
values are within a reasonable range when considering PRE or POST
separately, but when you put them together in a scatterplot you can see
that the values are off since we expect PRE and POST weights to be
somewhat similar.

\begin{itemize}
\tightlist
\item
  Two individuals have PRE weights that are \textless{} 100 pounds
  (bottom left side of plot).

  \begin{itemize}
  \tightlist
  \item
    There is a good chance that these weights may have been accidentally
    recorded as kg (kilograms) instead of in pounds.
  \end{itemize}
\item
  And there are 2 individuals with POST weights around 100-120 lbs, but
  for whom their PRE weights were 225-260 lbs.

  \begin{itemize}
  \tightlist
  \item
    There is a good chance that these two data points may have had a
    typo in the first number (e.g.~a weight of 110 should be 210).
  \end{itemize}
\item
  For this made-up dataset, it also appears that all 4 of these odd data
  points are Males. It is a good idea to explore other ``correlates''
  that may help identify underlying data collection issues.
\end{itemize}

Let's correct these values.

\begin{Shaded}
\begin{Highlighting}[]
\CommentTok{\# for WeightPRE \textless{} 100, convert kg to lbs}
\NormalTok{mydata\_corrected }\OtherTok{\textless{}{-}}\NormalTok{ mydata\_corrected }\SpecialCharTok{\%\textgreater{}\%}
  \FunctionTok{mutate}\NormalTok{(}\AttributeTok{WeightPRE\_corrected =} \FunctionTok{case\_when}\NormalTok{(}
\NormalTok{    (WeightPRE }\SpecialCharTok{\textless{}} \DecValTok{100}\NormalTok{) }\SpecialCharTok{\textasciitilde{}}\NormalTok{ WeightPRE }\SpecialCharTok{*} \FloatTok{2.20462}\NormalTok{,}
    \AttributeTok{.default =}\NormalTok{ WeightPRE}
\NormalTok{  ))}
\end{Highlighting}
\end{Shaded}

\begin{Shaded}
\begin{Highlighting}[]
\CommentTok{\# For WeightPOST, for}
\CommentTok{\# SubjectID 28, change WeightPOST=98 to 198}
\CommentTok{\# since this person\textquotesingle{}s WeightPRE was 230.}
\CommentTok{\# also fix SubjectID = 32, for}
\CommentTok{\# WeightPOST from 109 to 209 since}
\CommentTok{\# their WeightPRE was 260}

\NormalTok{mydata\_corrected }\OtherTok{\textless{}{-}}\NormalTok{ mydata\_corrected }\SpecialCharTok{\%\textgreater{}\%}
  \FunctionTok{mutate}\NormalTok{(}\AttributeTok{WeightPOST\_corrected =} \FunctionTok{case\_when}\NormalTok{(}
\NormalTok{    (SubjectID }\SpecialCharTok{==} \DecValTok{28}\NormalTok{) }\SpecialCharTok{\textasciitilde{}} \DecValTok{198}\NormalTok{,}
\NormalTok{    (SubjectID }\SpecialCharTok{==} \DecValTok{32}\NormalTok{) }\SpecialCharTok{\textasciitilde{}} \DecValTok{209}\NormalTok{,}
    \AttributeTok{.default =}\NormalTok{ WeightPOST}
\NormalTok{  ))}
\end{Highlighting}
\end{Shaded}

\newpage

Let's redo the plot with these corrected values - now the PRE and POST
weights look similar.

I've also added a ``reference line'' (in ``red'' color) to the plot
below. By adding the line ``Y = X'' we can also visualize which points
are above or below the line for people who gained or lost weight from
PRE-to-POST, respectively. It looks like most people lost weight - the
majority of the points are below the line where PRE \textgreater{} POST
weights.

I also:

\begin{itemize}
\tightlist
\item
  applied colors to each gender category,
\item
  applied shapes to each gender category,
\item
  changed the size of the points,
\item
  assigned custom colors for each gender category,

  \begin{itemize}
  \tightlist
  \item
    the colors are for the non-missing values
  \item
    if you want to see the person missing a gender, we have to
    specifically assign a color for NA using \texttt{na.value=}
  \end{itemize}
\item
  assigned custom shapes for each gender category,

  \begin{itemize}
  \tightlist
  \item
    the colors are for the non-missing values
  \item
    if you want to see the person missing a gender, we have to
    specifically assign a color for NA using \texttt{na.value=}
  \end{itemize}
\item
  also notice that I had to provide a custom label in the
  \texttt{labs()} for the shape and color legend - \textbf{the labels
  are the same for \texttt{color} and \texttt{shape}} so they will be in
  the same legend box. \emph{It is possible to assign the variables for
  color and shape to different variables.}
\end{itemize}

\begin{Shaded}
\begin{Highlighting}[]
\FunctionTok{ggplot}\NormalTok{(}\AttributeTok{data =}\NormalTok{ mydata\_corrected,}
       \FunctionTok{aes}\NormalTok{(}\AttributeTok{x =}\NormalTok{ WeightPRE\_corrected,}
           \AttributeTok{y =}\NormalTok{ WeightPOST\_corrected,}
           \AttributeTok{color =}\NormalTok{ GenderCoded.f,}
           \AttributeTok{shape =}\NormalTok{ GenderCoded.f)) }\SpecialCharTok{+}
  \FunctionTok{geom\_point}\NormalTok{(}\AttributeTok{size =} \DecValTok{2}\NormalTok{) }\SpecialCharTok{+}
  \FunctionTok{geom\_abline}\NormalTok{(}\AttributeTok{slope =} \DecValTok{1}\NormalTok{, }
              \AttributeTok{intercept =} \DecValTok{0}\NormalTok{,}
              \AttributeTok{color =} \StringTok{"red"}\NormalTok{) }\SpecialCharTok{+}
  \FunctionTok{scale\_shape\_manual}\NormalTok{(}\AttributeTok{values =} \FunctionTok{c}\NormalTok{(}\DecValTok{16}\NormalTok{, }\DecValTok{17}\NormalTok{),}
                     \AttributeTok{na.value =} \DecValTok{15}\NormalTok{) }\SpecialCharTok{+}
  \FunctionTok{scale\_color\_manual}\NormalTok{(}\AttributeTok{values =} \FunctionTok{c}\NormalTok{(}\StringTok{"blue"}\NormalTok{, }
                                \StringTok{"magenta"}\NormalTok{),}
                     \AttributeTok{na.value =} \StringTok{"grey30"}\NormalTok{) }\SpecialCharTok{+}
  \FunctionTok{xlab}\NormalTok{(}\StringTok{"Weight (in pounds) before program"}\NormalTok{) }\SpecialCharTok{+}
  \FunctionTok{ylab}\NormalTok{(}\StringTok{"Weight (in pounds) after program"}\NormalTok{) }\SpecialCharTok{+}
  \FunctionTok{labs}\NormalTok{(}
    \AttributeTok{title =} \StringTok{"Weights (in pounds) before and after"}\NormalTok{,}
    \AttributeTok{subtitle =} \StringTok{"Hypothetical Madeup mydata Dataset"}\NormalTok{,}
    \AttributeTok{color =} \StringTok{"Gender"}\NormalTok{,}
    \AttributeTok{shape =} \StringTok{"Gender"}
\NormalTok{  ) }
\end{Highlighting}
\end{Shaded}

\pandocbounded{\includegraphics[keepaspectratio]{module133_DataVis_files/figure-pdf/unnamed-chunk-19-1.pdf}}

\newpage

\subsubsection{\texorpdfstring{\texttt{ggplot2} -
Histogram}{ggplot2 - Histogram}}\label{ggplot2---histogram}

Let's make a histogram of \texttt{Age} and overlay a density curve like
we did above for the heights, but this time using the \texttt{ggplot2}
package functions.

The first step:

\begin{itemize}
\tightlist
\item
  specify the dataset \texttt{mydata\_corrected} and ``aesthetics''
  variable \texttt{x=Age} inside the \texttt{ggplot()} step
\item
  then add the geometric object \texttt{geom\_histogram()}
\end{itemize}

\begin{Shaded}
\begin{Highlighting}[]
\FunctionTok{ggplot}\NormalTok{(}\AttributeTok{data =}\NormalTok{ mydata\_corrected, }
       \FunctionTok{aes}\NormalTok{(}\AttributeTok{x =}\NormalTok{ Age)) }\SpecialCharTok{+}
  \FunctionTok{geom\_histogram}\NormalTok{()}
\end{Highlighting}
\end{Shaded}

\pandocbounded{\includegraphics[keepaspectratio]{module133_DataVis_files/figure-pdf/unnamed-chunk-20-1.pdf}}

\newpage

Let's add some color using \texttt{fill=} for the inside colors of the
bars and \texttt{color=} for the border color for the bars.

\begin{Shaded}
\begin{Highlighting}[]
\FunctionTok{ggplot}\NormalTok{(mydata\_corrected, }
       \FunctionTok{aes}\NormalTok{(}\AttributeTok{x =}\NormalTok{ Age)) }\SpecialCharTok{+}
  \FunctionTok{geom\_histogram}\NormalTok{(}\AttributeTok{fill =} \StringTok{"lightblue"}\NormalTok{,}
                 \AttributeTok{color =} \StringTok{"black"}\NormalTok{)}
\end{Highlighting}
\end{Shaded}

\pandocbounded{\includegraphics[keepaspectratio]{module133_DataVis_files/figure-pdf/unnamed-chunk-21-1.pdf}}

\newpage

To add the density curve, we need to do 2 things:

\begin{enumerate}
\def\labelenumi{\arabic{enumi}.}
\tightlist
\item
  Add an aesthetic \texttt{aes()} to change from counts (or frequencies)
  for the bars to probabilities. We can do this using the
  \texttt{after\_stat()} function.

  \begin{itemize}
  \tightlist
  \item
    Learn more by running
    \texttt{help(aes\_eval,\ package\ =\ "ggplot2")}.
  \end{itemize}
\item
  And then we can add the \texttt{geom\_density()} geometric object and
  add \texttt{color=} for the overlaid line color.
\end{enumerate}

And I also added some better labels to the axes, title and subtitle.

\begin{Shaded}
\begin{Highlighting}[]
\FunctionTok{ggplot}\NormalTok{(mydata\_corrected, }
       \FunctionTok{aes}\NormalTok{(}\AttributeTok{x =}\NormalTok{ Age,}
           \AttributeTok{y =} \FunctionTok{after\_stat}\NormalTok{(density))) }\SpecialCharTok{+}
  \FunctionTok{geom\_histogram}\NormalTok{(}\AttributeTok{fill =} \StringTok{"lightblue"}\NormalTok{,}
                 \AttributeTok{color =} \StringTok{"black"}\NormalTok{) }\SpecialCharTok{+}
  \FunctionTok{geom\_density}\NormalTok{(}\AttributeTok{color =} \StringTok{"red"}\NormalTok{) }\SpecialCharTok{+}
  \FunctionTok{xlab}\NormalTok{(}\StringTok{"Age (in years)"}\NormalTok{) }\SpecialCharTok{+}
  \FunctionTok{ylab}\NormalTok{(}\StringTok{"Proportion"}\NormalTok{) }\SpecialCharTok{+}
  \FunctionTok{labs}\NormalTok{(}
    \AttributeTok{title =} \StringTok{"Ages for Participants"}\NormalTok{,}
    \AttributeTok{subtitle =} \StringTok{"Hypothetical Madeup mydata Dataset"}
\NormalTok{  ) }
\end{Highlighting}
\end{Shaded}

\pandocbounded{\includegraphics[keepaspectratio]{module133_DataVis_files/figure-pdf/unnamed-chunk-22-1.pdf}}

\newpage

\subsubsection{\texorpdfstring{\texttt{ggplot2} - Boxplot (and
variations)}{ggplot2 - Boxplot (and variations)}}\label{ggplot2---boxplot-and-variations}

Let's look at the corrected PRE weights by SES.

\begin{Shaded}
\begin{Highlighting}[]
\FunctionTok{ggplot}\NormalTok{(}\AttributeTok{data =}\NormalTok{ mydata\_corrected,}
       \FunctionTok{aes}\NormalTok{(}\AttributeTok{x =}\NormalTok{ SES.f,}
           \AttributeTok{y =}\NormalTok{ WeightPRE\_corrected)) }\SpecialCharTok{+}
  \FunctionTok{geom\_boxplot}\NormalTok{()}
\end{Highlighting}
\end{Shaded}

\pandocbounded{\includegraphics[keepaspectratio]{module133_DataVis_files/figure-pdf/unnamed-chunk-23-1.pdf}}

\newpage

There is one person missing SES. So, let's filter the dataset and remake
the plot. Instead of creating another ``new'' dataset, we can use the
\texttt{dplyr} pipe \texttt{\%\textgreater{}\%} into our plotting
workflow as follows to filter out the missing SES before we make the
plot. Notice I can drop the \texttt{data\ =} in the \texttt{ggplot()}
step.

In the \texttt{filter()} step below, I used the \texttt{!} exclamation
point to indicate that we want to keep all rows for which \texttt{SES.f}
is NOT missing, by using \texttt{!is.na()}.

\begin{Shaded}
\begin{Highlighting}[]
\FunctionTok{library}\NormalTok{(dplyr)}

\NormalTok{mydata\_corrected }\SpecialCharTok{\%\textgreater{}\%}
  \FunctionTok{filter}\NormalTok{(}\SpecialCharTok{!}\FunctionTok{is.na}\NormalTok{(SES.f)) }\SpecialCharTok{\%\textgreater{}\%}
  \FunctionTok{ggplot}\NormalTok{(}\FunctionTok{aes}\NormalTok{(}\AttributeTok{x =}\NormalTok{ SES.f, }
             \AttributeTok{y =}\NormalTok{ WeightPRE\_corrected)) }\SpecialCharTok{+}
  \FunctionTok{geom\_boxplot}\NormalTok{()}
\end{Highlighting}
\end{Shaded}

\pandocbounded{\includegraphics[keepaspectratio]{module133_DataVis_files/figure-pdf/unnamed-chunk-24-1.pdf}}

\newpage

Let's add a fill color for the SES categories. Notice that a legend is
automatically added to the plot for the SES colors.

\begin{Shaded}
\begin{Highlighting}[]
\NormalTok{mydata\_corrected }\SpecialCharTok{\%\textgreater{}\%}
  \FunctionTok{filter}\NormalTok{(}\SpecialCharTok{!}\FunctionTok{is.na}\NormalTok{(SES.f)) }\SpecialCharTok{\%\textgreater{}\%}
  \FunctionTok{ggplot}\NormalTok{(}\FunctionTok{aes}\NormalTok{(}\AttributeTok{x =}\NormalTok{ SES.f, }
             \AttributeTok{y =}\NormalTok{ WeightPRE\_corrected,}
             \AttributeTok{fill =}\NormalTok{ SES.f)) }\SpecialCharTok{+}
  \FunctionTok{geom\_boxplot}\NormalTok{()}
\end{Highlighting}
\end{Shaded}

\pandocbounded{\includegraphics[keepaspectratio]{module133_DataVis_files/figure-pdf/unnamed-chunk-25-1.pdf}}

\newpage

And add better axis labels plus a title and subtitle.

\begin{Shaded}
\begin{Highlighting}[]
\NormalTok{mydata\_corrected }\SpecialCharTok{\%\textgreater{}\%}
  \FunctionTok{filter}\NormalTok{(}\SpecialCharTok{!}\FunctionTok{is.na}\NormalTok{(SES.f)) }\SpecialCharTok{\%\textgreater{}\%}
  \FunctionTok{ggplot}\NormalTok{(}\FunctionTok{aes}\NormalTok{(}\AttributeTok{x =}\NormalTok{ SES.f, }
             \AttributeTok{y =}\NormalTok{ WeightPRE\_corrected,}
             \AttributeTok{fill =}\NormalTok{ SES.f)) }\SpecialCharTok{+}
  \FunctionTok{geom\_boxplot}\NormalTok{() }\SpecialCharTok{+}
  \FunctionTok{xlab}\NormalTok{(}\StringTok{"Socio{-}Economic Status Categories"}\NormalTok{) }\SpecialCharTok{+}
  \FunctionTok{ylab}\NormalTok{(}\StringTok{"Weight (in pounds) before program"}\NormalTok{) }\SpecialCharTok{+}
  \FunctionTok{labs}\NormalTok{(}
    \AttributeTok{title =} \StringTok{"Weights by SES Categories"}\NormalTok{,}
    \AttributeTok{subtitle =} \StringTok{"Hypothetical Madeup mydata Dataset"}
\NormalTok{  ) }
\end{Highlighting}
\end{Shaded}

\pandocbounded{\includegraphics[keepaspectratio]{module133_DataVis_files/figure-pdf/unnamed-chunk-26-1.pdf}}

\newpage

\textbf{Add Another Layer with Points}

We can also add points on top of the boxplots using
\texttt{geom\_jitter()} AFTER using \texttt{geom\_boxplot}. If you
switch the order of these ``geom's'' you can specify whether the boxplot
is on top of the points or if the points are on top of the boxplots
(like we did here).

In \texttt{geom\_jitter()}, I also added \texttt{height=0} and
\texttt{width=.10} to adjust the amount of jitter in the vertical and
horizontal directions.

\begin{Shaded}
\begin{Highlighting}[]
\NormalTok{mydata\_corrected }\SpecialCharTok{\%\textgreater{}\%}
  \FunctionTok{filter}\NormalTok{(}\SpecialCharTok{!}\FunctionTok{is.na}\NormalTok{(SES.f)) }\SpecialCharTok{\%\textgreater{}\%}
  \FunctionTok{ggplot}\NormalTok{(}\FunctionTok{aes}\NormalTok{(}\AttributeTok{x =}\NormalTok{ SES.f, }
             \AttributeTok{y =}\NormalTok{ WeightPRE\_corrected,}
             \AttributeTok{fill =}\NormalTok{ SES.f)) }\SpecialCharTok{+}
  \FunctionTok{geom\_boxplot}\NormalTok{() }\SpecialCharTok{+}
  \FunctionTok{geom\_jitter}\NormalTok{(}\AttributeTok{height=}\DecValTok{0}\NormalTok{, }
              \AttributeTok{width=}\NormalTok{.}\DecValTok{10}\NormalTok{) }\SpecialCharTok{+}
  \FunctionTok{xlab}\NormalTok{(}\StringTok{"Socio{-}Economic Status Categories"}\NormalTok{) }\SpecialCharTok{+}
  \FunctionTok{ylab}\NormalTok{(}\StringTok{"Weight (in pounds) before program"}\NormalTok{) }\SpecialCharTok{+}
  \FunctionTok{labs}\NormalTok{(}
    \AttributeTok{title =} \StringTok{"Weights by SES Categories"}\NormalTok{,}
    \AttributeTok{subtitle =} \StringTok{"Hypothetical Madeup mydata Dataset"}\NormalTok{,}
    \AttributeTok{fill =} \StringTok{"SES Categories"}
\NormalTok{  ) }
\end{Highlighting}
\end{Shaded}

\pandocbounded{\includegraphics[keepaspectratio]{module133_DataVis_files/figure-pdf/unnamed-chunk-27-1.pdf}}

A couple more packages to look at the distribution of data points by
groups are:

\begin{itemize}
\tightlist
\item
  \texttt{beeswarm}

  \begin{itemize}
  \tightlist
  \item
    \href{https://r-graph-gallery.com/beeswarm.html}{R Graph Gallery
    Example of \texttt{beeswarm}}
  \item
    \href{https://cran.r-project.org/web/packages/beeswarm/index.html}{\texttt{beeswarm}
    on CRAN}
  \end{itemize}
\item
  \href{https://cran.r-project.org/web/packages/ggbeeswarm/index.html}{\texttt{ggbeeswarm}
  on CRAN}
\end{itemize}

\newpage

\textbf{Try Another Geom}

One of the cool things about \texttt{ggplot2} is the ability to easily
swap out \texttt{geom}'s. Let's try a violin plot which provides a
better idea of the shape of the underlying distributions that you don't
get with a simple boxplot. Change \texttt{geom\_boxplot()} to
\texttt{geom\_violin()}. I also added the \texttt{bw} argument to change
the ``bandwidth'' for how much smoothing is done. Try changing this
number and see what happens. Learn more by running
\texttt{help(geom\_violin,\ package\ =\ "ggplot2")}

\begin{Shaded}
\begin{Highlighting}[]
\NormalTok{mydata\_corrected }\SpecialCharTok{\%\textgreater{}\%}
  \FunctionTok{filter}\NormalTok{(}\SpecialCharTok{!}\FunctionTok{is.na}\NormalTok{(SES.f)) }\SpecialCharTok{\%\textgreater{}\%}
  \FunctionTok{ggplot}\NormalTok{(}\FunctionTok{aes}\NormalTok{(}\AttributeTok{x =}\NormalTok{ SES.f, }
             \AttributeTok{y =}\NormalTok{ WeightPRE\_corrected,}
             \AttributeTok{fill =}\NormalTok{ SES.f)) }\SpecialCharTok{+}
  \FunctionTok{geom\_violin}\NormalTok{(}\AttributeTok{bw=}\DecValTok{10}\NormalTok{) }\SpecialCharTok{+}
  \FunctionTok{xlab}\NormalTok{(}\StringTok{"Socio{-}Economic Status Categories"}\NormalTok{) }\SpecialCharTok{+}
  \FunctionTok{ylab}\NormalTok{(}\StringTok{"Weight (in pounds) before program"}\NormalTok{) }\SpecialCharTok{+}
  \FunctionTok{labs}\NormalTok{(}
    \AttributeTok{title =} \StringTok{"Weights by SES Categories"}\NormalTok{,}
    \AttributeTok{subtitle =} \StringTok{"Hypothetical Madeup mydata Dataset"}
\NormalTok{  ) }
\end{Highlighting}
\end{Shaded}

\pandocbounded{\includegraphics[keepaspectratio]{module133_DataVis_files/figure-pdf/unnamed-chunk-28-1.pdf}}

\newpage

\subsubsection{\texorpdfstring{\texttt{ggplot2} -
Barchart}{ggplot2 - Barchart}}\label{ggplot2---barchart}

Let's make a simple barchart for \texttt{SES.f} after filtering out the
\texttt{NA}s.

\begin{Shaded}
\begin{Highlighting}[]
\NormalTok{mydata\_corrected }\SpecialCharTok{\%\textgreater{}\%}
  \FunctionTok{filter}\NormalTok{(}\SpecialCharTok{!}\FunctionTok{is.na}\NormalTok{(SES.f)) }\SpecialCharTok{\%\textgreater{}\%}
  \FunctionTok{ggplot}\NormalTok{(}\FunctionTok{aes}\NormalTok{(}\AttributeTok{x =}\NormalTok{ SES.f)) }\SpecialCharTok{+}
  \FunctionTok{geom\_bar}\NormalTok{()}
\end{Highlighting}
\end{Shaded}

\pandocbounded{\includegraphics[keepaspectratio]{module133_DataVis_files/figure-pdf/unnamed-chunk-29-1.pdf}}

\newpage

Let's also make a clustered barplot of \texttt{SES.f} by
\texttt{GenderCoded.f}. Let's also filter out the \texttt{NA}s from
\texttt{GenderCoded.f} as well.

To add the 2nd grouping or clustering variable, we add \texttt{fill=} to
the aesthetics and then add \texttt{position\ =\ "dodge"} for
\texttt{geom\_bar()} to see the colors side by side instead of stacked.

\begin{Shaded}
\begin{Highlighting}[]
\NormalTok{mydata\_corrected }\SpecialCharTok{\%\textgreater{}\%}
  \FunctionTok{filter}\NormalTok{(}\SpecialCharTok{!}\FunctionTok{is.na}\NormalTok{(SES.f)) }\SpecialCharTok{\%\textgreater{}\%}
  \FunctionTok{filter}\NormalTok{(}\SpecialCharTok{!}\FunctionTok{is.na}\NormalTok{(GenderCoded.f)) }\SpecialCharTok{\%\textgreater{}\%}
  \FunctionTok{ggplot}\NormalTok{(}\FunctionTok{aes}\NormalTok{(}\AttributeTok{x =}\NormalTok{ SES.f,}
             \AttributeTok{fill =}\NormalTok{ GenderCoded.f)) }\SpecialCharTok{+}
  \FunctionTok{geom\_bar}\NormalTok{(}\AttributeTok{position =} \StringTok{"dodge"}\NormalTok{)}
\end{Highlighting}
\end{Shaded}

\pandocbounded{\includegraphics[keepaspectratio]{module133_DataVis_files/figure-pdf/unnamed-chunk-30-1.pdf}}

\newpage

Let's also add custom colors and better labels. A few notes:

\begin{itemize}
\tightlist
\item
  \texttt{fill} controls the interior filled color for the bars
\item
  \texttt{color} controls the border color of the bars
\item
  \texttt{scale\_fill\_manual()} is where you add custom colors
\end{itemize}

\begin{Shaded}
\begin{Highlighting}[]
\NormalTok{mydata\_corrected }\SpecialCharTok{\%\textgreater{}\%}
  \FunctionTok{filter}\NormalTok{(}\SpecialCharTok{!}\FunctionTok{is.na}\NormalTok{(SES.f)) }\SpecialCharTok{\%\textgreater{}\%}
  \FunctionTok{filter}\NormalTok{(}\SpecialCharTok{!}\FunctionTok{is.na}\NormalTok{(GenderCoded.f)) }\SpecialCharTok{\%\textgreater{}\%}
  \FunctionTok{ggplot}\NormalTok{(}\FunctionTok{aes}\NormalTok{(}\AttributeTok{x =}\NormalTok{ SES.f,}
             \AttributeTok{fill =}\NormalTok{ GenderCoded.f)) }\SpecialCharTok{+}
  \FunctionTok{geom\_bar}\NormalTok{(}\AttributeTok{position =} \StringTok{"dodge"}\NormalTok{, }
           \AttributeTok{color =} \StringTok{"black"}\NormalTok{) }\SpecialCharTok{+}
  \FunctionTok{scale\_fill\_manual}\NormalTok{(}\AttributeTok{values =} \FunctionTok{c}\NormalTok{(}\StringTok{"blue"}\NormalTok{, }
                               \StringTok{"magenta"}\NormalTok{)) }\SpecialCharTok{+}
  \FunctionTok{xlab}\NormalTok{(}\StringTok{"Socio{-}Economic Status Categories"}\NormalTok{) }\SpecialCharTok{+}
  \FunctionTok{ylab}\NormalTok{(}\StringTok{"Frequency"}\NormalTok{) }\SpecialCharTok{+}
  \FunctionTok{labs}\NormalTok{(}
    \AttributeTok{title =} \StringTok{"Frequencies of SES Categories by Gender"}\NormalTok{,}
    \AttributeTok{subtitle =} \StringTok{"Hypothetical Madeup mydata Dataset"}\NormalTok{,}
    \AttributeTok{fill =} \StringTok{"Gender"}
\NormalTok{  )}
\end{Highlighting}
\end{Shaded}

\pandocbounded{\includegraphics[keepaspectratio]{module133_DataVis_files/figure-pdf/unnamed-chunk-31-1.pdf}}

\newpage

\subsubsection{\texorpdfstring{\texttt{ggplot2} - Errorbar
plots}{ggplot2 - Errorbar plots}}\label{ggplot2---errorbar-plots}

\textbf{Barplot with Error Bars}

Suppose instead of boxplots for looking at the differences in heights by
gender, let's make a plot of the mean heights by gender with error bars
added to reflect the 95\% confidence intervals for these group means.

First let's compute the means using the \texttt{mean()} function and
we'll pull the 95\% confidence interval limits from the
\texttt{t.test()} output for a one-sample t-test \emph{(see details
below)}.

Save these outputs into another small dataset called \texttt{dt}.

\textbf{Let's take a look at the \texttt{t.test()} output.}

We can run a one-sample t-test to test whether the mean of the Heights
is different from 0. This will result in a p-value for this hypothesis
test. The \texttt{t.test()} output can also be used to evaluate to see
whether or not the 95\% confidence interval contains 0 or not.

\[H_0: \mu_{height} = 0\] versus

\[H_a: \mu_{height} \neq 0\]

\begin{Shaded}
\begin{Highlighting}[]
\CommentTok{\# run one{-}sample t{-}test and save the output}
\CommentTok{\# into an object called tt1}
\NormalTok{tt1 }\OtherTok{\textless{}{-}} \FunctionTok{t.test}\NormalTok{(mydata\_corrected}\SpecialCharTok{$}\NormalTok{Height\_corrected, }
              \AttributeTok{conf.level =} \FloatTok{0.95}\NormalTok{)}
\end{Highlighting}
\end{Shaded}

If we print the \texttt{tt1} object to the console, we get the
abbreviated t-test results which gives us the t-test statistic, p-value,
the mean of the heights and the 95\% confidence interval for that mean
(which does not include 0).

\begin{Shaded}
\begin{Highlighting}[]
\NormalTok{tt1}
\end{Highlighting}
\end{Shaded}

\begin{verbatim}

    One Sample t-test

data:  mydata_corrected$Height_corrected
t = 61.664, df = 18, p-value < 2.2e-16
alternative hypothesis: true mean is not equal to 0
95 percent confidence interval:
 5.658315 6.057475
sample estimates:
mean of x 
 5.857895 
\end{verbatim}

But the \texttt{tt1} t-test output object actually has a bunch of
details stored inside it. Let's look at the structure of the
\texttt{tt1} t-test object:

\begin{Shaded}
\begin{Highlighting}[]
\FunctionTok{str}\NormalTok{(tt1)}
\end{Highlighting}
\end{Shaded}

\begin{verbatim}
List of 10
 $ statistic  : Named num 61.7
  ..- attr(*, "names")= chr "t"
 $ parameter  : Named num 18
  ..- attr(*, "names")= chr "df"
 $ p.value    : num 2.13e-22
 $ conf.int   : num [1:2] 5.66 6.06
  ..- attr(*, "conf.level")= num 0.95
 $ estimate   : Named num 5.86
  ..- attr(*, "names")= chr "mean of x"
 $ null.value : Named num 0
  ..- attr(*, "names")= chr "mean"
 $ stderr     : num 0.095
 $ alternative: chr "two.sided"
 $ method     : chr "One Sample t-test"
 $ data.name  : chr "mydata_corrected$Height_corrected"
 - attr(*, "class")= chr "htest"
\end{verbatim}

We can select elements of this t-test object just like we select
variables out of a dataset using the \texttt{\$} dollar sign selector.
Let's take a look at the \texttt{conf.int} part of the t-test object for
the 95\% confidence interval limits.

\begin{Shaded}
\begin{Highlighting}[]
\NormalTok{tt1}\SpecialCharTok{$}\NormalTok{conf.int}
\end{Highlighting}
\end{Shaded}

\begin{verbatim}
[1] 5.658315 6.057475
attr(,"conf.level")
[1] 0.95
\end{verbatim}

We can further pull out each limit separately using the \texttt{{[}{]}}
square brackets to pull specifically the first element \texttt{{[}1{]}}
for the lower limit of the 95\% confidence interval and the second
element \texttt{{[}2{]}} for the upper limit of the 95\% confidence
interval.

\begin{Shaded}
\begin{Highlighting}[]
\NormalTok{tt1}\SpecialCharTok{$}\NormalTok{conf.int[}\DecValTok{1}\NormalTok{]}
\end{Highlighting}
\end{Shaded}

\begin{verbatim}
[1] 5.658315
\end{verbatim}

\begin{Shaded}
\begin{Highlighting}[]
\NormalTok{tt1}\SpecialCharTok{$}\NormalTok{conf.int[}\DecValTok{2}\NormalTok{]}
\end{Highlighting}
\end{Shaded}

\begin{verbatim}
[1] 6.057475
\end{verbatim}

So, we will save these components from the t-test output inside the
\texttt{mutate} step in the code below to make sure we capture the lower
confidence interval \texttt{lci} and upper confidence interval
\texttt{uci} for each gender category which we will later use for making
our error bars.

\begin{Shaded}
\begin{Highlighting}[]
\CommentTok{\# capture the means of the correct heights }
\CommentTok{\# and get the 95\% confidence intervals}
\CommentTok{\# upper bound and lower bound by gender}
\CommentTok{\# filter out the missing GenderCoded.f}
\NormalTok{dt }\OtherTok{\textless{}{-}}\NormalTok{ mydata\_corrected }\SpecialCharTok{\%\textgreater{}\%}
  \FunctionTok{filter}\NormalTok{(}\SpecialCharTok{!}\FunctionTok{is.na}\NormalTok{(GenderCoded.f)) }\SpecialCharTok{\%\textgreater{}\%}
\NormalTok{  dplyr}\SpecialCharTok{::}\FunctionTok{group\_by}\NormalTok{(GenderCoded.f)}\SpecialCharTok{\%\textgreater{}\%}
\NormalTok{  dplyr}\SpecialCharTok{::}\FunctionTok{summarise}\NormalTok{(}
    \AttributeTok{mean =} \FunctionTok{mean}\NormalTok{(Height\_corrected, }\AttributeTok{na.rm =} \ConstantTok{TRUE}\NormalTok{),}
    \AttributeTok{lci =} \FunctionTok{t.test}\NormalTok{(Height\_corrected, }
                 \AttributeTok{conf.level =} \FloatTok{0.95}\NormalTok{)}\SpecialCharTok{$}\NormalTok{conf.int[}\DecValTok{1}\NormalTok{],}
    \AttributeTok{uci =} \FunctionTok{t.test}\NormalTok{(Height\_corrected, }
                 \AttributeTok{conf.level =} \FloatTok{0.95}\NormalTok{)}\SpecialCharTok{$}\NormalTok{conf.int[}\DecValTok{2}\NormalTok{])}
\NormalTok{dt}
\end{Highlighting}
\end{Shaded}

\begin{verbatim}
# A tibble: 2 x 4
  GenderCoded.f  mean   lci   uci
  <fct>         <dbl> <dbl> <dbl>
1 Male           6.1   5.86  6.34
2 Female         5.59  5.30  5.88
\end{verbatim}

\newpage

Use this small dataset \texttt{dt} with the means and 95\% confidence
intervals limits to make this plot.

\begin{Shaded}
\begin{Highlighting}[]
\FunctionTok{ggplot}\NormalTok{(}\AttributeTok{data =}\NormalTok{ dt) }\SpecialCharTok{+}
  \FunctionTok{geom\_bar}\NormalTok{(}\FunctionTok{aes}\NormalTok{(}\AttributeTok{x =}\NormalTok{ GenderCoded.f, }
               \AttributeTok{y =}\NormalTok{ mean, }
               \AttributeTok{fill =}\NormalTok{ GenderCoded.f), }
           \AttributeTok{color =} \StringTok{"black"}\NormalTok{,}
           \AttributeTok{stat=}\StringTok{"identity"}\NormalTok{) }\SpecialCharTok{+}
  \FunctionTok{scale\_fill\_manual}\NormalTok{(}\AttributeTok{values =} \FunctionTok{c}\NormalTok{(}\StringTok{"blue"}\NormalTok{, }
                               \StringTok{"magenta"}\NormalTok{)) }\SpecialCharTok{+} 
  \FunctionTok{geom\_errorbar}\NormalTok{(}\FunctionTok{aes}\NormalTok{(}\AttributeTok{x =}\NormalTok{ GenderCoded.f, }
                    \AttributeTok{ymin =}\NormalTok{ lci, }
                    \AttributeTok{ymax =}\NormalTok{ uci), }
                \AttributeTok{width =} \FloatTok{0.4}\NormalTok{, }
                \AttributeTok{color =}\StringTok{"black"}\NormalTok{, }
                \AttributeTok{size =} \DecValTok{1}\NormalTok{) }\SpecialCharTok{+}
  \FunctionTok{xlab}\NormalTok{(}\StringTok{"Gender"}\NormalTok{) }\SpecialCharTok{+}
  \FunctionTok{ylab}\NormalTok{(}\StringTok{"Mean Height (in decimal feet)"}\NormalTok{) }\SpecialCharTok{+}
  \FunctionTok{labs}\NormalTok{(}
    \AttributeTok{title =} \StringTok{"Average Heights by Gender"}\NormalTok{,}
    \AttributeTok{subtitle =} \StringTok{"Hypothetical Madeup mydata Dataset"}\NormalTok{,}
    \AttributeTok{caption =} \StringTok{"Error Bars Represent 95\% Confidence Intervals"}\NormalTok{,}
    \AttributeTok{fill =} \StringTok{"Gender"}
\NormalTok{  )}
\end{Highlighting}
\end{Shaded}

\pandocbounded{\includegraphics[keepaspectratio]{module133_DataVis_files/figure-pdf/unnamed-chunk-38-1.pdf}}

\newpage

\textbf{Lineplot with Points and Error Bars}

We can also remove the bars and just create a line plot connecting the
points for the means with the error bars shown. I set
\texttt{size=1.5}to make the lines a little thicker in the plot.

\begin{Shaded}
\begin{Highlighting}[]
\FunctionTok{ggplot}\NormalTok{(}\AttributeTok{data =}\NormalTok{ dt) }\SpecialCharTok{+}
  \FunctionTok{geom\_point}\NormalTok{(}\FunctionTok{aes}\NormalTok{(}\AttributeTok{x =}\NormalTok{ GenderCoded.f, }
                 \AttributeTok{y =}\NormalTok{ mean,}
                 \AttributeTok{color =}\NormalTok{ GenderCoded.f),}
             \AttributeTok{size =} \DecValTok{3}\NormalTok{) }\SpecialCharTok{+} 
  \FunctionTok{geom\_errorbar}\NormalTok{(}\FunctionTok{aes}\NormalTok{(}\AttributeTok{x =}\NormalTok{ GenderCoded.f, }
                    \AttributeTok{ymin =}\NormalTok{ lci, }
                    \AttributeTok{ymax =}\NormalTok{ uci,}
                    \AttributeTok{color =}\NormalTok{ GenderCoded.f), }
                \AttributeTok{width =} \FloatTok{0.4}\NormalTok{, }
                \AttributeTok{size =} \FloatTok{1.5}\NormalTok{) }\SpecialCharTok{+}
  \FunctionTok{geom\_line}\NormalTok{(}\FunctionTok{aes}\NormalTok{(}\AttributeTok{x =}\NormalTok{ GenderCoded.f, }
                \AttributeTok{y =}\NormalTok{ mean), }
            \AttributeTok{group =} \DecValTok{1}\NormalTok{,}
            \AttributeTok{size =} \FloatTok{1.5}\NormalTok{,}
            \AttributeTok{color =} \StringTok{"black"}\NormalTok{) }\SpecialCharTok{+}
  \FunctionTok{scale\_color\_manual}\NormalTok{(}\AttributeTok{values =} \FunctionTok{c}\NormalTok{(}\StringTok{"blue"}\NormalTok{, }
                               \StringTok{"magenta"}\NormalTok{)) }\SpecialCharTok{+} 
  \FunctionTok{xlab}\NormalTok{(}\StringTok{"Gender"}\NormalTok{) }\SpecialCharTok{+}
  \FunctionTok{ylab}\NormalTok{(}\StringTok{"Mean Height (in decimal feet)"}\NormalTok{) }\SpecialCharTok{+}
  \FunctionTok{labs}\NormalTok{(}
    \AttributeTok{title =} \StringTok{"Average Heights by Gender"}\NormalTok{,}
    \AttributeTok{subtitle =} \StringTok{"Hypothetical Madeup mydata Dataset"}\NormalTok{,}
    \AttributeTok{caption =} \StringTok{"Error Bars Represent 95\% Confidence Intervals"}\NormalTok{,}
    \AttributeTok{color =} \StringTok{"Gender"}
\NormalTok{  )}
\end{Highlighting}
\end{Shaded}

\pandocbounded{\includegraphics[keepaspectratio]{module133_DataVis_files/figure-pdf/unnamed-chunk-39-1.pdf}}

\newpage

\subsubsection{\texorpdfstring{\texttt{ggplot2} - Lollipop
plots}{ggplot2 - Lollipop plots}}\label{ggplot2---lollipop-plots}

Another plot that can be useful to help visualize changes between two
time points,like PRE-to-POST changes is using a ``lollipop'' plot which
utilizes the \texttt{geom\_segment()} to create line segments with
points (the lollipops) on each end.

The plot below was inspired by the code example at
\url{https://r-graph-gallery.com/303-lollipop-plot-with-2-values.html}.

Let's look at the corrected Weights PRE to POST- sorted by their
starting PRE weights.

\begin{Shaded}
\begin{Highlighting}[]
\CommentTok{\# sort data by WeightPRE\_corrected ascending}
\NormalTok{data }\OtherTok{\textless{}{-}}\NormalTok{ mydata\_corrected }\SpecialCharTok{\%\textgreater{}\%}
  \FunctionTok{rowwise}\NormalTok{() }\SpecialCharTok{\%\textgreater{}\%}
  \FunctionTok{arrange}\NormalTok{(WeightPRE\_corrected) }\SpecialCharTok{\%\textgreater{}\%}
  \FunctionTok{mutate}\NormalTok{(}\AttributeTok{SubjectID =} \FunctionTok{factor}\NormalTok{(SubjectID, SubjectID))}

\CommentTok{\# Plot}
\FunctionTok{ggplot}\NormalTok{(data) }\SpecialCharTok{+}
  \FunctionTok{geom\_segment}\NormalTok{(}\FunctionTok{aes}\NormalTok{(}\AttributeTok{x =}\NormalTok{ SubjectID,}
                   \AttributeTok{xend =}\NormalTok{ SubjectID,}
                   \AttributeTok{y =}\NormalTok{ WeightPRE\_corrected,}
                   \AttributeTok{yend =}\NormalTok{ WeightPOST\_corrected), }
    \AttributeTok{color =} \StringTok{"grey30"}\NormalTok{) }\SpecialCharTok{+}
  \FunctionTok{geom\_point}\NormalTok{(}\FunctionTok{aes}\NormalTok{(}\AttributeTok{x =}\NormalTok{ SubjectID, }
                 \AttributeTok{y =}\NormalTok{ WeightPRE\_corrected,}
                 \AttributeTok{color =} \StringTok{"WeightPRE\_corrected"}\NormalTok{),}
             \AttributeTok{size =} \DecValTok{3}\NormalTok{) }\SpecialCharTok{+}
  \FunctionTok{geom\_point}\NormalTok{(}\FunctionTok{aes}\NormalTok{(}\AttributeTok{x =}\NormalTok{ SubjectID, }
                 \AttributeTok{y =}\NormalTok{ WeightPOST\_corrected,}
                 \AttributeTok{color =} \StringTok{"WeightPOST\_corrected"}\NormalTok{),}
             \AttributeTok{size =} \DecValTok{3}\NormalTok{) }\SpecialCharTok{+}
  \FunctionTok{scale\_color\_manual}\NormalTok{(}
    \AttributeTok{labels =} \FunctionTok{c}\NormalTok{(}\StringTok{"PRE"}\NormalTok{, }\StringTok{"POST"}\NormalTok{),}
    \AttributeTok{values =} \FunctionTok{c}\NormalTok{(}\StringTok{"coral"}\NormalTok{,}\StringTok{"darkblue"}\NormalTok{),}
    \AttributeTok{guide  =} \FunctionTok{guide\_legend}\NormalTok{(), }
    \AttributeTok{name   =} \StringTok{"Group"}\NormalTok{) }\SpecialCharTok{+}
  \FunctionTok{coord\_flip}\NormalTok{() }\SpecialCharTok{+}
  \FunctionTok{theme}\NormalTok{(}\AttributeTok{legend.position =} \StringTok{"bottom"}\NormalTok{) }\SpecialCharTok{+}
  \FunctionTok{xlab}\NormalTok{(}\StringTok{"Subject IDs"}\NormalTok{) }\SpecialCharTok{+}
  \FunctionTok{ylab}\NormalTok{(}\StringTok{"Weight Change (in pounds) PRE to POST"}\NormalTok{)}
\end{Highlighting}
\end{Shaded}

\pandocbounded{\includegraphics[keepaspectratio]{module133_DataVis_files/figure-pdf/unnamed-chunk-40-1.pdf}}

\begin{center}\rule{0.5\linewidth}{0.5pt}\end{center}

\newpage

\subsection{3. Other Graphics Packages to
Know}\label{other-graphics-packages-to-know}

\subsubsection{Save plot objects and reuse/rearrange
them}\label{save-plot-objects-and-reuserearrange-them}

Once a ``chunk'' of \texttt{ggplot} code is run, technically a
\texttt{ggplot2} plot object is created. We can save and reuse these
objects to create composite figures.

For example, let's create the scatterplot, histogram and clustered
barplot and save each into three separate plot objects \texttt{p1},
\texttt{p2}, and \texttt{p3}.

\begin{Shaded}
\begin{Highlighting}[]
\CommentTok{\# make the scatterplot, save as p1}

\NormalTok{p1 }\OtherTok{\textless{}{-}} \FunctionTok{ggplot}\NormalTok{(}
  \AttributeTok{data =}\NormalTok{ mydata\_corrected,}
  \FunctionTok{aes}\NormalTok{(}
    \AttributeTok{x =}\NormalTok{ WeightPRE\_corrected,}
    \AttributeTok{y =}\NormalTok{ WeightPOST\_corrected,}
    \AttributeTok{color =}\NormalTok{ GenderCoded.f,}
    \AttributeTok{shape =}\NormalTok{ GenderCoded.f}
\NormalTok{  )) }\SpecialCharTok{+}
  \FunctionTok{geom\_point}\NormalTok{(}\AttributeTok{size =} \DecValTok{2}\NormalTok{) }\SpecialCharTok{+}
  \FunctionTok{geom\_abline}\NormalTok{(}\AttributeTok{slope =} \DecValTok{1}\NormalTok{,}
              \AttributeTok{intercept =} \DecValTok{0}\NormalTok{,}
              \AttributeTok{color =} \StringTok{"red"}\NormalTok{) }\SpecialCharTok{+}
  \FunctionTok{scale\_shape\_manual}\NormalTok{(}\AttributeTok{values =} \FunctionTok{c}\NormalTok{(}\DecValTok{16}\NormalTok{, }\DecValTok{17}\NormalTok{), }\AttributeTok{na.value =} \DecValTok{15}\NormalTok{) }\SpecialCharTok{+}
  \FunctionTok{scale\_color\_manual}\NormalTok{(}\AttributeTok{values =} \FunctionTok{c}\NormalTok{(}\StringTok{"blue"}\NormalTok{, }\StringTok{"magenta"}\NormalTok{),}
                     \AttributeTok{na.value =} \StringTok{"grey30"}\NormalTok{) }\SpecialCharTok{+}
  \FunctionTok{xlab}\NormalTok{(}\StringTok{"Weight (in pounds) before program"}\NormalTok{) }\SpecialCharTok{+}
  \FunctionTok{ylab}\NormalTok{(}\StringTok{"Weight (in pounds) after program"}\NormalTok{) }\SpecialCharTok{+}
  \FunctionTok{labs}\NormalTok{(}
    \AttributeTok{title =} \StringTok{"Weights (in pounds) before and after"}\NormalTok{,}
    \AttributeTok{subtitle =} \StringTok{"Hypothetical Madeup mydata Dataset"}\NormalTok{,}
    \AttributeTok{color =} \StringTok{"Gender"}\NormalTok{,}
    \AttributeTok{shape =} \StringTok{"Gender"}
\NormalTok{  )}
\end{Highlighting}
\end{Shaded}

\begin{Shaded}
\begin{Highlighting}[]
\CommentTok{\# make the histogram, save as p2}

\NormalTok{p2 }\OtherTok{\textless{}{-}} \FunctionTok{ggplot}\NormalTok{(mydata\_corrected, }
             \FunctionTok{aes}\NormalTok{(}\AttributeTok{x =}\NormalTok{ Age, }
                 \AttributeTok{y =} \FunctionTok{after\_stat}\NormalTok{(density))) }\SpecialCharTok{+}
  \FunctionTok{geom\_histogram}\NormalTok{(}\AttributeTok{fill =} \StringTok{"lightblue"}\NormalTok{, }\AttributeTok{color =} \StringTok{"black"}\NormalTok{) }\SpecialCharTok{+}
  \FunctionTok{geom\_density}\NormalTok{(}\AttributeTok{color =} \StringTok{"red"}\NormalTok{) }\SpecialCharTok{+}
  \FunctionTok{xlab}\NormalTok{(}\StringTok{"Age (in years)"}\NormalTok{) }\SpecialCharTok{+}
  \FunctionTok{ylab}\NormalTok{(}\StringTok{"Proportion"}\NormalTok{) }\SpecialCharTok{+}
  \FunctionTok{labs}\NormalTok{(}\AttributeTok{title =} \StringTok{"Ages for Participants"}\NormalTok{, }
       \AttributeTok{subtitle =} \StringTok{"Hypothetical Madeup mydata Dataset"}\NormalTok{)}
\end{Highlighting}
\end{Shaded}

\begin{Shaded}
\begin{Highlighting}[]
\CommentTok{\# make the barplot, save as p3}

\NormalTok{p3 }\OtherTok{\textless{}{-}}\NormalTok{ mydata\_corrected }\SpecialCharTok{\%\textgreater{}\%}
  \FunctionTok{filter}\NormalTok{(}\SpecialCharTok{!}\FunctionTok{is.na}\NormalTok{(SES.f)) }\SpecialCharTok{\%\textgreater{}\%}
  \FunctionTok{filter}\NormalTok{(}\SpecialCharTok{!}\FunctionTok{is.na}\NormalTok{(GenderCoded.f)) }\SpecialCharTok{\%\textgreater{}\%}
  \FunctionTok{ggplot}\NormalTok{(}\FunctionTok{aes}\NormalTok{(}\AttributeTok{x =}\NormalTok{ SES.f,}
             \AttributeTok{fill =}\NormalTok{ GenderCoded.f)) }\SpecialCharTok{+}
  \FunctionTok{geom\_bar}\NormalTok{(}\AttributeTok{position =} \StringTok{"dodge"}\NormalTok{, }
           \AttributeTok{color =} \StringTok{"black"}\NormalTok{) }\SpecialCharTok{+}
  \FunctionTok{scale\_fill\_manual}\NormalTok{(}\AttributeTok{values =} \FunctionTok{c}\NormalTok{(}\StringTok{"blue"}\NormalTok{, }
                               \StringTok{"magenta"}\NormalTok{)) }\SpecialCharTok{+}
  \FunctionTok{xlab}\NormalTok{(}\StringTok{"Socio{-}Economic Status Categories"}\NormalTok{) }\SpecialCharTok{+}
  \FunctionTok{ylab}\NormalTok{(}\StringTok{"Frequency"}\NormalTok{) }\SpecialCharTok{+}
  \FunctionTok{labs}\NormalTok{(}
    \AttributeTok{title =} \StringTok{"Frequencies of SES Categories by Gender"}\NormalTok{,}
    \AttributeTok{subtitle =} \StringTok{"Hypothetical Madeup mydata Dataset"}\NormalTok{,}
    \AttributeTok{fill =} \StringTok{"Gender"}
\NormalTok{  )}
\end{Highlighting}
\end{Shaded}

\newpage

\paragraph{\texorpdfstring{\textbf{\texttt{patchwork}
package}}{patchwork package}}\label{patchwork-package}

After making and saving each of the \texttt{ggplot2} plot objects above,
we can arrange them into a new composite plot. The
\href{https://cran.r-project.org/web/packages/patchwork/index.html}{\texttt{patchwork}
package} is really good for making these composite figures.

Learn more at
\url{https://patchwork.data-imaginist.com/articles/patchwork.html}

\begin{Shaded}
\begin{Highlighting}[]
\CommentTok{\# load patchwork package}
\FunctionTok{library}\NormalTok{(patchwork)}

\CommentTok{\# put p1 and p2 side{-}by{-}side}
\CommentTok{\# and put both of these on top of p3}
\NormalTok{(p1 }\SpecialCharTok{+}\NormalTok{ p2) }\SpecialCharTok{/}\NormalTok{ p3}
\end{Highlighting}
\end{Shaded}

\pandocbounded{\includegraphics[keepaspectratio]{module133_DataVis_files/figure-pdf/unnamed-chunk-44-1.pdf}}

\newpage

\paragraph{\texorpdfstring{\textbf{\texttt{ggpubr} package and
\texttt{ggarrange()}
function}}{ggpubr package and ggarrange() function}}\label{ggpubr-package-and-ggarrange-function}

Another package that is useful for arrange plot objects into composite
plots is the
\href{https://cran.r-project.org/web/packages/ggpubr/}{\texttt{ggpubr}
package} with the
\href{https://rpkgs.datanovia.com/ggpubr/reference/ggarrange.html}{\texttt{ggarrange()}
function}.

\begin{Shaded}
\begin{Highlighting}[]
\CommentTok{\# load ggpubr package}
\FunctionTok{library}\NormalTok{(ggpubr)}

\CommentTok{\# use ggarrange twice}
\CommentTok{\# put p1 and p2 side by side}
\CommentTok{\# then put on top of p3}
\FunctionTok{ggarrange}\NormalTok{(}
  \FunctionTok{ggarrange}\NormalTok{(p1, p2, }\AttributeTok{widths =} \FunctionTok{c}\NormalTok{(}\DecValTok{1}\NormalTok{, }\DecValTok{1}\NormalTok{)),}
\NormalTok{  p3, }\AttributeTok{nrow =} \DecValTok{2}\NormalTok{, }\AttributeTok{ncol =} \DecValTok{1}\NormalTok{)}
\end{Highlighting}
\end{Shaded}

\pandocbounded{\includegraphics[keepaspectratio]{module133_DataVis_files/figure-pdf/unnamed-chunk-45-1.pdf}}

\newpage

\subsubsection{\texorpdfstring{\texttt{GGally} package and
\texttt{ggpairs()}
function}{GGally package and ggpairs() function}}\label{ggally-package-and-ggpairs-function}

If you would like to make a scatterplot matrix to look at the
associations (correlations) between multiple numeric variables at the
same time, the
\href{https://ggobi.github.io/ggally/articles/ggpairs.html}{\texttt{GGally::ggpairs()}
function} is useful.

In the plot below, we can see the 2-dimensional scatterplots between the
heights and weights at PRE and POST. The plot also provides the
Pearson's correlations for all of the pairwise associations between each
combination of 2 variables.

I also added a ``best fit'' linear line by adding
\texttt{lower\ =\ list(continuous\ =\ "smooth")}.

\begin{Shaded}
\begin{Highlighting}[]
\FunctionTok{library}\NormalTok{(GGally)}

\FunctionTok{ggpairs}\NormalTok{(mydata\_corrected,}
        \AttributeTok{columns =} \FunctionTok{c}\NormalTok{(}\StringTok{"Height\_corrected"}\NormalTok{, }
                    \StringTok{"WeightPRE\_corrected"}\NormalTok{, }
                    \StringTok{"WeightPOST\_corrected"}\NormalTok{),}
        \AttributeTok{lower =} \FunctionTok{list}\NormalTok{(}\AttributeTok{continuous =} \StringTok{"smooth"}\NormalTok{))}
\end{Highlighting}
\end{Shaded}

\pandocbounded{\includegraphics[keepaspectratio]{module133_DataVis_files/figure-pdf/unnamed-chunk-46-1.pdf}}

\newpage

What is really cool about this plotting function is the easy way to add
in a 3rd variable like gender to see if these correlations (and
scatterplots) change by gender (i.e.~does gender moderate the
associations?). Notice that we get separate lines for each gender and we
get the correlations for each gender as well.

If you look at the correlations by gender between
\texttt{Height\_corrected} and \texttt{WeightPOST\_corrected} the
correlation for Males was 0.692 and for Females was 0.913, so it does
look like the correlation is stronger for the Females than the Males.
For this made-up dataset, this doesn't matter. But this approach is a
good way to start exploring your data for moderating effects.

\begin{Shaded}
\begin{Highlighting}[]
\FunctionTok{ggpairs}\NormalTok{(mydata\_corrected,}
        \AttributeTok{mapping =} \FunctionTok{aes}\NormalTok{(}\AttributeTok{color =}\NormalTok{ GenderCoded.f),}
        \AttributeTok{columns =} \FunctionTok{c}\NormalTok{(}\StringTok{"Height\_corrected"}\NormalTok{, }
                    \StringTok{"WeightPRE\_corrected"}\NormalTok{, }
                    \StringTok{"WeightPOST\_corrected"}\NormalTok{),}
        \AttributeTok{lower =} \FunctionTok{list}\NormalTok{(}\AttributeTok{continuous =} \StringTok{"smooth"}\NormalTok{))}
\end{Highlighting}
\end{Shaded}

\pandocbounded{\includegraphics[keepaspectratio]{module133_DataVis_files/figure-pdf/unnamed-chunk-47-1.pdf}}

\newpage

And if we add \texttt{GenderCoded.f} to the list of variable columns to
be included, we now also get bar charts for the frequencies of each
gender, histograms of each variable for each gender, boxplots for each
variable by each gender, along with the density curves by gender and the
correlation matrix.

\begin{Shaded}
\begin{Highlighting}[]
\FunctionTok{ggpairs}\NormalTok{(mydata\_corrected,}
        \AttributeTok{mapping =} \FunctionTok{aes}\NormalTok{(}\AttributeTok{color =}\NormalTok{ GenderCoded.f),}
        \AttributeTok{columns =} \FunctionTok{c}\NormalTok{(}\StringTok{"GenderCoded.f"}\NormalTok{,}
                    \StringTok{"Height\_corrected"}\NormalTok{, }
                    \StringTok{"WeightPRE\_corrected"}\NormalTok{, }
                    \StringTok{"WeightPOST\_corrected"}\NormalTok{),}
        \AttributeTok{lower =} \FunctionTok{list}\NormalTok{(}\AttributeTok{continuous =} \StringTok{"smooth"}\NormalTok{))}
\end{Highlighting}
\end{Shaded}

\pandocbounded{\includegraphics[keepaspectratio]{module133_DataVis_files/figure-pdf/unnamed-chunk-48-1.pdf}}

\newpage

\subsubsection{\texorpdfstring{Visualize Categorical Data with
\texttt{vcd}
package}{Visualize Categorical Data with vcd package}}\label{visualize-categorical-data-with-vcd-package}

The
\href{https://cran.r-project.org/web/packages/vcd/index.html}{\texttt{vcd}
package} for ``visualizing categorical data'' has been around for over
20 years!! It is a helpful package for visualizing multiple categorical
variables at once.

Let's visualize the relative proportions of gender and SES using the
\texttt{vcd::mosaic()} function.

\begin{Shaded}
\begin{Highlighting}[]
\FunctionTok{library}\NormalTok{(vcd)}

\NormalTok{vcd}\SpecialCharTok{::}\FunctionTok{mosaic}\NormalTok{(GenderCoded.f }\SpecialCharTok{\textasciitilde{}}\NormalTok{ SES.f, }
            \AttributeTok{data =}\NormalTok{ mydata\_corrected,}
            \AttributeTok{gp =} \FunctionTok{gpar}\NormalTok{(}\AttributeTok{fill =} \FunctionTok{c}\NormalTok{(}\StringTok{"gray"}\NormalTok{,}\StringTok{"dark magenta"}\NormalTok{)),}
            \AttributeTok{main =} \StringTok{"Gender and SES"}\NormalTok{,}
\NormalTok{            )}
\end{Highlighting}
\end{Shaded}

\pandocbounded{\includegraphics[keepaspectratio]{module133_DataVis_files/figure-pdf/unnamed-chunk-49-1.pdf}}

\newpage

\subsubsection{\texorpdfstring{Example of an animated graph with
\texttt{gganimate}}{Example of an animated graph with gganimate}}\label{example-of-an-animated-graph-with-gganimate}

Learn how to make an animated figure with the \texttt{gganimate} package
at \url{https://gganimate.com/}. The animation demo shown below is of
the relationship between Life Expectancy and GDP (gross domestic
product) per capita in 142 countries over 12 years from 1952 to 2007.

The animated figure is viewable in the HTML website at
\url{https://melindahiggins2000.github.io/emory_tidal_Rlectures/module133_DataVis.html}.
In the PDF file only a static view is shown for the single year ``1976''
from the \href{https://jennybc.github.io/gapminder/}{\texttt{gapminder}}
dataset and R package.

\includegraphics[width=0.95\linewidth,height=\textheight,keepaspectratio]{myannimategif.png}

\newpage

\subsubsection{\texorpdfstring{Interactive Graphics with
\texttt{plotly}}{Interactive Graphics with plotly}}\label{interactive-graphics-with-plotly}

Another cool package is the \texttt{plotly} graphics package which
allows for active interaction with the plot. This works in the RStudio
environment (viewing in the ``Viewer'' window) and from within an HTML
formatted document rendered from Rmarkdown.

Learn more at \url{https://plotly-r.com/}.

Here is an interactive version (on this website for the HTML version) of
the side-by-side boxplots below - with a horizontal orientation.
\emph{This plot will NOT be interactive in the PDF document but the
figure will be shown.}

\begin{Shaded}
\begin{Highlighting}[]
\FunctionTok{library}\NormalTok{(plotly)}
\NormalTok{fig }\OtherTok{\textless{}{-}} \FunctionTok{plot\_ly}\NormalTok{(mydata\_corrected, }
               \AttributeTok{x =} \SpecialCharTok{\textasciitilde{}}\NormalTok{WeightPRE\_corrected, }
               \AttributeTok{color =} \SpecialCharTok{\textasciitilde{}}\NormalTok{SES.f, }
               \AttributeTok{type =} \StringTok{"box"}\NormalTok{,}
               \AttributeTok{orientation =} \StringTok{"h"}\NormalTok{)}
\NormalTok{fig}
\end{Highlighting}
\end{Shaded}

\pandocbounded{\includegraphics[keepaspectratio]{module133_DataVis_files/figure-pdf/unnamed-chunk-52-1.pdf}}

\subsection{4. Summary Tables with
Graphics}\label{summary-tables-with-graphics}

There are a number of other packages that allow you to insert small
graphs or figures inside of a table.

The \href{https://r-graph-gallery.com/table.html}{R Graph Gallery} has a
nice summary of table packages with these features. Some of these
packages include:

\begin{itemize}
\tightlist
\item
  \href{https://cran.r-project.org/web/packages/gtExtras/index.html}{\texttt{gtExtras}}
\item
  \href{https://cran.r-project.org/web/packages/huxtable/index.html}{\texttt{huxtable}}
\item
  \href{https://ardata-fr.github.io/flextable-book/index.html\#help-and-resources}{\texttt{flextable}}
\item
  \href{https://cran.r-project.org/web/packages/skimr/index.html}{\texttt{skimr}}
\item
  \href{https://modelsummary.com/vignettes/datasummary.html}{\texttt{modelsummary}}
\item
  \href{https://cran.r-project.org/web/packages/rhandsontable/index.html}{\texttt{rhandsontable}}
\end{itemize}

As an example, let's look at the code at
\url{https://r-graph-gallery.com/368-plotting-in-cells-with-gtextras.html}
for making a table of summary statistics with a plot overview including
a list of categories and percentage of missing data for that variable.

\begin{Shaded}
\begin{Highlighting}[]
\FunctionTok{library}\NormalTok{(gtExtras)}

\NormalTok{mydata\_corrected }\SpecialCharTok{\%\textgreater{}\%}
  \FunctionTok{select}\NormalTok{(Height\_corrected,}
\NormalTok{         WeightPRE\_corrected,}
\NormalTok{         WeightPOST\_corrected,}
\NormalTok{         GenderCoded.f,}
\NormalTok{         SES.f) }\SpecialCharTok{\%\textgreater{}\%}
  \FunctionTok{gt\_plt\_summary}\NormalTok{()}
\end{Highlighting}
\end{Shaded}

\pandocbounded{\includegraphics[keepaspectratio]{module133_DataVis_files/figure-pdf/unnamed-chunk-54-1.png}}

\begin{center}\rule{0.5\linewidth}{0.5pt}\end{center}

\newpage

\subsection{5. Other Places to Get Help and Get
Started}\label{other-places-to-get-help-and-get-started}

\begin{itemize}
\tightlist
\item
  See the summary of graphics resources at
  \href{./additionalResources.html\#r-graphics}{Additional Resources - R
  Graphics}
\end{itemize}

\begin{center}\rule{0.5\linewidth}{0.5pt}\end{center}

\subsection{R Code For This Module}\label{r-code-for-this-module}

\begin{itemize}
\tightlist
\item
  \href{}{\texttt{module\_133.R}}
\end{itemize}

\subsection{References}\label{references}

\phantomsection\label{refs}
\begin{CSLReferences}{1}{0}
\bibitem[\citeproctext]{ref-R-gt}
Iannone, Richard, Joe Cheng, Barret Schloerke, Ellis Hughes, Alexandra
Lauer, JooYoung Seo, Ken Brevoort, and Olivier Roy. 2024. \emph{Gt:
Easily Create Presentation-Ready Display Tables}.
\url{https://gt.rstudio.com}.

\bibitem[\citeproctext]{ref-R-ggpubr}
Kassambara, Alboukadel. 2023. \emph{Ggpubr: Ggplot2 Based Publication
Ready Plots}. \url{https://rpkgs.datanovia.com/ggpubr/}.

\bibitem[\citeproctext]{ref-vcd2006}
Meyer, David, Achim Zeileis, and Kurt Hornik. 2006. {``The Strucplot
Framework: Visualizing Multi-Way Contingency Tables with Vcd.''}
\emph{Journal of Statistical Software} 17 (3): 1--48.
\url{https://doi.org/10.18637/jss.v017.i03}.

\bibitem[\citeproctext]{ref-R-vcd}
Meyer, David, Achim Zeileis, Kurt Hornik, and Michael Friendly. 2023.
\emph{Vcd: Visualizing Categorical Data}.
\url{https://CRAN.R-project.org/package=vcd}.

\bibitem[\citeproctext]{ref-R-gtExtras}
Mock, Thomas. 2024. \emph{gtExtras: Extending Gt for Beautiful HTML
Tables}. \url{https://github.com/jthomasmock/gtExtras}.

\bibitem[\citeproctext]{ref-R-patchwork}
Pedersen, Thomas Lin. 2024. \emph{Patchwork: The Composer of Plots}.
\url{https://patchwork.data-imaginist.com}.

\bibitem[\citeproctext]{ref-R-base}
R Core Team. 2024. \emph{R: A Language and Environment for Statistical
Computing}. Vienna, Austria: R Foundation for Statistical Computing.
\url{https://www.R-project.org/}.

\bibitem[\citeproctext]{ref-R-GGally}
Schloerke, Barret, Di Cook, Joseph Larmarange, Francois Briatte, Moritz
Marbach, Edwin Thoen, Amos Elberg, and Jason Crowley. 2024.
\emph{GGally: Extension to Ggplot2}.
\url{https://ggobi.github.io/ggally/}.

\bibitem[\citeproctext]{ref-plotly2020}
Sievert, Carson. 2020. \emph{Interactive Web-Based Data Visualization
with r, Plotly, and Shiny}. Chapman; Hall/CRC.
\url{https://plotly-r.com}.

\bibitem[\citeproctext]{ref-R-plotly}
Sievert, Carson, Chris Parmer, Toby Hocking, Scott Chamberlain, Karthik
Ram, Marianne Corvellec, and Pedro Despouy. 2024. \emph{Plotly: Create
Interactive Web Graphics via Plotly.js}. \url{https://plotly-r.com}.

\bibitem[\citeproctext]{ref-ggplot22016}
Wickham, Hadley. 2016. \emph{Ggplot2: Elegant Graphics for Data
Analysis}. Springer-Verlag New York.
\url{https://ggplot2.tidyverse.org}.

\bibitem[\citeproctext]{ref-R-ggplot2}
Wickham, Hadley, Winston Chang, Lionel Henry, Thomas Lin Pedersen,
Kohske Takahashi, Claus Wilke, Kara Woo, Hiroaki Yutani, Dewey
Dunnington, and Teun van den Brand. 2024. \emph{Ggplot2: Create Elegant
Data Visualisations Using the Grammar of Graphics}.
\url{https://ggplot2.tidyverse.org}.

\bibitem[\citeproctext]{ref-R-dplyr}
Wickham, Hadley, Romain François, Lionel Henry, Kirill Müller, and Davis
Vaughan. 2023. \emph{Dplyr: A Grammar of Data Manipulation}.
\url{https://dplyr.tidyverse.org}.

\bibitem[\citeproctext]{ref-vcd2007}
Zeileis, Achim, David Meyer, and Kurt Hornik. 2007. {``Residual-Based
Shadings for Visualizing (Conditional) Independence.''} \emph{Journal of
Computational and Graphical Statistics} 16 (3): 507--25.
\url{https://doi.org/10.1198/106186007X237856}.

\end{CSLReferences}

\subsection{Other Helpful Resources}\label{other-helpful-resources}

\href{./additionalResources.html}{\textbf{Other Helpful Resources}}




\end{document}
